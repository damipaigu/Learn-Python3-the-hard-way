
% Default to the notebook output style

    


% Inherit from the specified cell style.




    
\documentclass[11pt]{article}

    
    
    \usepackage[T1]{fontenc}
    % Nicer default font (+ math font) than Computer Modern for most use cases
    \usepackage{mathpazo}

    % Basic figure setup, for now with no caption control since it's done
    % automatically by Pandoc (which extracts ![](path) syntax from Markdown).
    \usepackage{graphicx}
    % We will generate all images so they have a width \maxwidth. This means
    % that they will get their normal width if they fit onto the page, but
    % are scaled down if they would overflow the margins.
    \makeatletter
    \def\maxwidth{\ifdim\Gin@nat@width>\linewidth\linewidth
    \else\Gin@nat@width\fi}
    \makeatother
    \let\Oldincludegraphics\includegraphics
    % Set max figure width to be 80% of text width, for now hardcoded.
    \renewcommand{\includegraphics}[1]{\Oldincludegraphics[width=.8\maxwidth]{#1}}
    % Ensure that by default, figures have no caption (until we provide a
    % proper Figure object with a Caption API and a way to capture that
    % in the conversion process - todo).
    \usepackage{caption}
    \DeclareCaptionLabelFormat{nolabel}{}
    \captionsetup{labelformat=nolabel}

    \usepackage{adjustbox} % Used to constrain images to a maximum size 
    \usepackage{xcolor} % Allow colors to be defined
    \usepackage{enumerate} % Needed for markdown enumerations to work
    \usepackage{geometry} % Used to adjust the document margins
    \usepackage{amsmath} % Equations
    \usepackage{amssymb} % Equations
    \usepackage{textcomp} % defines textquotesingle
    % Hack from http://tex.stackexchange.com/a/47451/13684:
    \AtBeginDocument{%
        \def\PYZsq{\textquotesingle}% Upright quotes in Pygmentized code
    }
    \usepackage{upquote} % Upright quotes for verbatim code
    \usepackage{eurosym} % defines \euro
    \usepackage[mathletters]{ucs} % Extended unicode (utf-8) support
    \usepackage[utf8x]{inputenc} % Allow utf-8 characters in the tex document
    \usepackage{fancyvrb} % verbatim replacement that allows latex
    \usepackage{grffile} % extends the file name processing of package graphics 
                         % to support a larger range 
    % The hyperref package gives us a pdf with properly built
    % internal navigation ('pdf bookmarks' for the table of contents,
    % internal cross-reference links, web links for URLs, etc.)
    \usepackage{hyperref}
    \usepackage{longtable} % longtable support required by pandoc >1.10
    \usepackage{booktabs}  % table support for pandoc > 1.12.2
    \usepackage[inline]{enumitem} % IRkernel/repr support (it uses the enumerate* environment)
    \usepackage[normalem]{ulem} % ulem is needed to support strikethroughs (\sout)
                                % normalem makes italics be italics, not underlines
    

    
    
    % Colors for the hyperref package
    \definecolor{urlcolor}{rgb}{0,.145,.698}
    \definecolor{linkcolor}{rgb}{.71,0.21,0.01}
    \definecolor{citecolor}{rgb}{.12,.54,.11}

    % ANSI colors
    \definecolor{ansi-black}{HTML}{3E424D}
    \definecolor{ansi-black-intense}{HTML}{282C36}
    \definecolor{ansi-red}{HTML}{E75C58}
    \definecolor{ansi-red-intense}{HTML}{B22B31}
    \definecolor{ansi-green}{HTML}{00A250}
    \definecolor{ansi-green-intense}{HTML}{007427}
    \definecolor{ansi-yellow}{HTML}{DDB62B}
    \definecolor{ansi-yellow-intense}{HTML}{B27D12}
    \definecolor{ansi-blue}{HTML}{208FFB}
    \definecolor{ansi-blue-intense}{HTML}{0065CA}
    \definecolor{ansi-magenta}{HTML}{D160C4}
    \definecolor{ansi-magenta-intense}{HTML}{A03196}
    \definecolor{ansi-cyan}{HTML}{60C6C8}
    \definecolor{ansi-cyan-intense}{HTML}{258F8F}
    \definecolor{ansi-white}{HTML}{C5C1B4}
    \definecolor{ansi-white-intense}{HTML}{A1A6B2}

    % commands and environments needed by pandoc snippets
    % extracted from the output of `pandoc -s`
    \providecommand{\tightlist}{%
      \setlength{\itemsep}{0pt}\setlength{\parskip}{0pt}}
    \DefineVerbatimEnvironment{Highlighting}{Verbatim}{commandchars=\\\{\}}
    % Add ',fontsize=\small' for more characters per line
    \newenvironment{Shaded}{}{}
    \newcommand{\KeywordTok}[1]{\textcolor[rgb]{0.00,0.44,0.13}{\textbf{{#1}}}}
    \newcommand{\DataTypeTok}[1]{\textcolor[rgb]{0.56,0.13,0.00}{{#1}}}
    \newcommand{\DecValTok}[1]{\textcolor[rgb]{0.25,0.63,0.44}{{#1}}}
    \newcommand{\BaseNTok}[1]{\textcolor[rgb]{0.25,0.63,0.44}{{#1}}}
    \newcommand{\FloatTok}[1]{\textcolor[rgb]{0.25,0.63,0.44}{{#1}}}
    \newcommand{\CharTok}[1]{\textcolor[rgb]{0.25,0.44,0.63}{{#1}}}
    \newcommand{\StringTok}[1]{\textcolor[rgb]{0.25,0.44,0.63}{{#1}}}
    \newcommand{\CommentTok}[1]{\textcolor[rgb]{0.38,0.63,0.69}{\textit{{#1}}}}
    \newcommand{\OtherTok}[1]{\textcolor[rgb]{0.00,0.44,0.13}{{#1}}}
    \newcommand{\AlertTok}[1]{\textcolor[rgb]{1.00,0.00,0.00}{\textbf{{#1}}}}
    \newcommand{\FunctionTok}[1]{\textcolor[rgb]{0.02,0.16,0.49}{{#1}}}
    \newcommand{\RegionMarkerTok}[1]{{#1}}
    \newcommand{\ErrorTok}[1]{\textcolor[rgb]{1.00,0.00,0.00}{\textbf{{#1}}}}
    \newcommand{\NormalTok}[1]{{#1}}
    
    % Additional commands for more recent versions of Pandoc
    \newcommand{\ConstantTok}[1]{\textcolor[rgb]{0.53,0.00,0.00}{{#1}}}
    \newcommand{\SpecialCharTok}[1]{\textcolor[rgb]{0.25,0.44,0.63}{{#1}}}
    \newcommand{\VerbatimStringTok}[1]{\textcolor[rgb]{0.25,0.44,0.63}{{#1}}}
    \newcommand{\SpecialStringTok}[1]{\textcolor[rgb]{0.73,0.40,0.53}{{#1}}}
    \newcommand{\ImportTok}[1]{{#1}}
    \newcommand{\DocumentationTok}[1]{\textcolor[rgb]{0.73,0.13,0.13}{\textit{{#1}}}}
    \newcommand{\AnnotationTok}[1]{\textcolor[rgb]{0.38,0.63,0.69}{\textbf{\textit{{#1}}}}}
    \newcommand{\CommentVarTok}[1]{\textcolor[rgb]{0.38,0.63,0.69}{\textbf{\textit{{#1}}}}}
    \newcommand{\VariableTok}[1]{\textcolor[rgb]{0.10,0.09,0.49}{{#1}}}
    \newcommand{\ControlFlowTok}[1]{\textcolor[rgb]{0.00,0.44,0.13}{\textbf{{#1}}}}
    \newcommand{\OperatorTok}[1]{\textcolor[rgb]{0.40,0.40,0.40}{{#1}}}
    \newcommand{\BuiltInTok}[1]{{#1}}
    \newcommand{\ExtensionTok}[1]{{#1}}
    \newcommand{\PreprocessorTok}[1]{\textcolor[rgb]{0.74,0.48,0.00}{{#1}}}
    \newcommand{\AttributeTok}[1]{\textcolor[rgb]{0.49,0.56,0.16}{{#1}}}
    \newcommand{\InformationTok}[1]{\textcolor[rgb]{0.38,0.63,0.69}{\textbf{\textit{{#1}}}}}
    \newcommand{\WarningTok}[1]{\textcolor[rgb]{0.38,0.63,0.69}{\textbf{\textit{{#1}}}}}
    
    
    % Define a nice break command that doesn't care if a line doesn't already
    % exist.
    \def\br{\hspace*{\fill} \\* }
    % Math Jax compatability definitions
    \def\gt{>}
    \def\lt{<}
    % Document parameters
    \title{Python3 Official Tutorial Notebook}
    
    
    

    % Pygments definitions
    
\makeatletter
\def\PY@reset{\let\PY@it=\relax \let\PY@bf=\relax%
    \let\PY@ul=\relax \let\PY@tc=\relax%
    \let\PY@bc=\relax \let\PY@ff=\relax}
\def\PY@tok#1{\csname PY@tok@#1\endcsname}
\def\PY@toks#1+{\ifx\relax#1\empty\else%
    \PY@tok{#1}\expandafter\PY@toks\fi}
\def\PY@do#1{\PY@bc{\PY@tc{\PY@ul{%
    \PY@it{\PY@bf{\PY@ff{#1}}}}}}}
\def\PY#1#2{\PY@reset\PY@toks#1+\relax+\PY@do{#2}}

\expandafter\def\csname PY@tok@w\endcsname{\def\PY@tc##1{\textcolor[rgb]{0.73,0.73,0.73}{##1}}}
\expandafter\def\csname PY@tok@c\endcsname{\let\PY@it=\textit\def\PY@tc##1{\textcolor[rgb]{0.25,0.50,0.50}{##1}}}
\expandafter\def\csname PY@tok@cp\endcsname{\def\PY@tc##1{\textcolor[rgb]{0.74,0.48,0.00}{##1}}}
\expandafter\def\csname PY@tok@k\endcsname{\let\PY@bf=\textbf\def\PY@tc##1{\textcolor[rgb]{0.00,0.50,0.00}{##1}}}
\expandafter\def\csname PY@tok@kp\endcsname{\def\PY@tc##1{\textcolor[rgb]{0.00,0.50,0.00}{##1}}}
\expandafter\def\csname PY@tok@kt\endcsname{\def\PY@tc##1{\textcolor[rgb]{0.69,0.00,0.25}{##1}}}
\expandafter\def\csname PY@tok@o\endcsname{\def\PY@tc##1{\textcolor[rgb]{0.40,0.40,0.40}{##1}}}
\expandafter\def\csname PY@tok@ow\endcsname{\let\PY@bf=\textbf\def\PY@tc##1{\textcolor[rgb]{0.67,0.13,1.00}{##1}}}
\expandafter\def\csname PY@tok@nb\endcsname{\def\PY@tc##1{\textcolor[rgb]{0.00,0.50,0.00}{##1}}}
\expandafter\def\csname PY@tok@nf\endcsname{\def\PY@tc##1{\textcolor[rgb]{0.00,0.00,1.00}{##1}}}
\expandafter\def\csname PY@tok@nc\endcsname{\let\PY@bf=\textbf\def\PY@tc##1{\textcolor[rgb]{0.00,0.00,1.00}{##1}}}
\expandafter\def\csname PY@tok@nn\endcsname{\let\PY@bf=\textbf\def\PY@tc##1{\textcolor[rgb]{0.00,0.00,1.00}{##1}}}
\expandafter\def\csname PY@tok@ne\endcsname{\let\PY@bf=\textbf\def\PY@tc##1{\textcolor[rgb]{0.82,0.25,0.23}{##1}}}
\expandafter\def\csname PY@tok@nv\endcsname{\def\PY@tc##1{\textcolor[rgb]{0.10,0.09,0.49}{##1}}}
\expandafter\def\csname PY@tok@no\endcsname{\def\PY@tc##1{\textcolor[rgb]{0.53,0.00,0.00}{##1}}}
\expandafter\def\csname PY@tok@nl\endcsname{\def\PY@tc##1{\textcolor[rgb]{0.63,0.63,0.00}{##1}}}
\expandafter\def\csname PY@tok@ni\endcsname{\let\PY@bf=\textbf\def\PY@tc##1{\textcolor[rgb]{0.60,0.60,0.60}{##1}}}
\expandafter\def\csname PY@tok@na\endcsname{\def\PY@tc##1{\textcolor[rgb]{0.49,0.56,0.16}{##1}}}
\expandafter\def\csname PY@tok@nt\endcsname{\let\PY@bf=\textbf\def\PY@tc##1{\textcolor[rgb]{0.00,0.50,0.00}{##1}}}
\expandafter\def\csname PY@tok@nd\endcsname{\def\PY@tc##1{\textcolor[rgb]{0.67,0.13,1.00}{##1}}}
\expandafter\def\csname PY@tok@s\endcsname{\def\PY@tc##1{\textcolor[rgb]{0.73,0.13,0.13}{##1}}}
\expandafter\def\csname PY@tok@sd\endcsname{\let\PY@it=\textit\def\PY@tc##1{\textcolor[rgb]{0.73,0.13,0.13}{##1}}}
\expandafter\def\csname PY@tok@si\endcsname{\let\PY@bf=\textbf\def\PY@tc##1{\textcolor[rgb]{0.73,0.40,0.53}{##1}}}
\expandafter\def\csname PY@tok@se\endcsname{\let\PY@bf=\textbf\def\PY@tc##1{\textcolor[rgb]{0.73,0.40,0.13}{##1}}}
\expandafter\def\csname PY@tok@sr\endcsname{\def\PY@tc##1{\textcolor[rgb]{0.73,0.40,0.53}{##1}}}
\expandafter\def\csname PY@tok@ss\endcsname{\def\PY@tc##1{\textcolor[rgb]{0.10,0.09,0.49}{##1}}}
\expandafter\def\csname PY@tok@sx\endcsname{\def\PY@tc##1{\textcolor[rgb]{0.00,0.50,0.00}{##1}}}
\expandafter\def\csname PY@tok@m\endcsname{\def\PY@tc##1{\textcolor[rgb]{0.40,0.40,0.40}{##1}}}
\expandafter\def\csname PY@tok@gh\endcsname{\let\PY@bf=\textbf\def\PY@tc##1{\textcolor[rgb]{0.00,0.00,0.50}{##1}}}
\expandafter\def\csname PY@tok@gu\endcsname{\let\PY@bf=\textbf\def\PY@tc##1{\textcolor[rgb]{0.50,0.00,0.50}{##1}}}
\expandafter\def\csname PY@tok@gd\endcsname{\def\PY@tc##1{\textcolor[rgb]{0.63,0.00,0.00}{##1}}}
\expandafter\def\csname PY@tok@gi\endcsname{\def\PY@tc##1{\textcolor[rgb]{0.00,0.63,0.00}{##1}}}
\expandafter\def\csname PY@tok@gr\endcsname{\def\PY@tc##1{\textcolor[rgb]{1.00,0.00,0.00}{##1}}}
\expandafter\def\csname PY@tok@ge\endcsname{\let\PY@it=\textit}
\expandafter\def\csname PY@tok@gs\endcsname{\let\PY@bf=\textbf}
\expandafter\def\csname PY@tok@gp\endcsname{\let\PY@bf=\textbf\def\PY@tc##1{\textcolor[rgb]{0.00,0.00,0.50}{##1}}}
\expandafter\def\csname PY@tok@go\endcsname{\def\PY@tc##1{\textcolor[rgb]{0.53,0.53,0.53}{##1}}}
\expandafter\def\csname PY@tok@gt\endcsname{\def\PY@tc##1{\textcolor[rgb]{0.00,0.27,0.87}{##1}}}
\expandafter\def\csname PY@tok@err\endcsname{\def\PY@bc##1{\setlength{\fboxsep}{0pt}\fcolorbox[rgb]{1.00,0.00,0.00}{1,1,1}{\strut ##1}}}
\expandafter\def\csname PY@tok@kc\endcsname{\let\PY@bf=\textbf\def\PY@tc##1{\textcolor[rgb]{0.00,0.50,0.00}{##1}}}
\expandafter\def\csname PY@tok@kd\endcsname{\let\PY@bf=\textbf\def\PY@tc##1{\textcolor[rgb]{0.00,0.50,0.00}{##1}}}
\expandafter\def\csname PY@tok@kn\endcsname{\let\PY@bf=\textbf\def\PY@tc##1{\textcolor[rgb]{0.00,0.50,0.00}{##1}}}
\expandafter\def\csname PY@tok@kr\endcsname{\let\PY@bf=\textbf\def\PY@tc##1{\textcolor[rgb]{0.00,0.50,0.00}{##1}}}
\expandafter\def\csname PY@tok@bp\endcsname{\def\PY@tc##1{\textcolor[rgb]{0.00,0.50,0.00}{##1}}}
\expandafter\def\csname PY@tok@fm\endcsname{\def\PY@tc##1{\textcolor[rgb]{0.00,0.00,1.00}{##1}}}
\expandafter\def\csname PY@tok@vc\endcsname{\def\PY@tc##1{\textcolor[rgb]{0.10,0.09,0.49}{##1}}}
\expandafter\def\csname PY@tok@vg\endcsname{\def\PY@tc##1{\textcolor[rgb]{0.10,0.09,0.49}{##1}}}
\expandafter\def\csname PY@tok@vi\endcsname{\def\PY@tc##1{\textcolor[rgb]{0.10,0.09,0.49}{##1}}}
\expandafter\def\csname PY@tok@vm\endcsname{\def\PY@tc##1{\textcolor[rgb]{0.10,0.09,0.49}{##1}}}
\expandafter\def\csname PY@tok@sa\endcsname{\def\PY@tc##1{\textcolor[rgb]{0.73,0.13,0.13}{##1}}}
\expandafter\def\csname PY@tok@sb\endcsname{\def\PY@tc##1{\textcolor[rgb]{0.73,0.13,0.13}{##1}}}
\expandafter\def\csname PY@tok@sc\endcsname{\def\PY@tc##1{\textcolor[rgb]{0.73,0.13,0.13}{##1}}}
\expandafter\def\csname PY@tok@dl\endcsname{\def\PY@tc##1{\textcolor[rgb]{0.73,0.13,0.13}{##1}}}
\expandafter\def\csname PY@tok@s2\endcsname{\def\PY@tc##1{\textcolor[rgb]{0.73,0.13,0.13}{##1}}}
\expandafter\def\csname PY@tok@sh\endcsname{\def\PY@tc##1{\textcolor[rgb]{0.73,0.13,0.13}{##1}}}
\expandafter\def\csname PY@tok@s1\endcsname{\def\PY@tc##1{\textcolor[rgb]{0.73,0.13,0.13}{##1}}}
\expandafter\def\csname PY@tok@mb\endcsname{\def\PY@tc##1{\textcolor[rgb]{0.40,0.40,0.40}{##1}}}
\expandafter\def\csname PY@tok@mf\endcsname{\def\PY@tc##1{\textcolor[rgb]{0.40,0.40,0.40}{##1}}}
\expandafter\def\csname PY@tok@mh\endcsname{\def\PY@tc##1{\textcolor[rgb]{0.40,0.40,0.40}{##1}}}
\expandafter\def\csname PY@tok@mi\endcsname{\def\PY@tc##1{\textcolor[rgb]{0.40,0.40,0.40}{##1}}}
\expandafter\def\csname PY@tok@il\endcsname{\def\PY@tc##1{\textcolor[rgb]{0.40,0.40,0.40}{##1}}}
\expandafter\def\csname PY@tok@mo\endcsname{\def\PY@tc##1{\textcolor[rgb]{0.40,0.40,0.40}{##1}}}
\expandafter\def\csname PY@tok@ch\endcsname{\let\PY@it=\textit\def\PY@tc##1{\textcolor[rgb]{0.25,0.50,0.50}{##1}}}
\expandafter\def\csname PY@tok@cm\endcsname{\let\PY@it=\textit\def\PY@tc##1{\textcolor[rgb]{0.25,0.50,0.50}{##1}}}
\expandafter\def\csname PY@tok@cpf\endcsname{\let\PY@it=\textit\def\PY@tc##1{\textcolor[rgb]{0.25,0.50,0.50}{##1}}}
\expandafter\def\csname PY@tok@c1\endcsname{\let\PY@it=\textit\def\PY@tc##1{\textcolor[rgb]{0.25,0.50,0.50}{##1}}}
\expandafter\def\csname PY@tok@cs\endcsname{\let\PY@it=\textit\def\PY@tc##1{\textcolor[rgb]{0.25,0.50,0.50}{##1}}}

\def\PYZbs{\char`\\}
\def\PYZus{\char`\_}
\def\PYZob{\char`\{}
\def\PYZcb{\char`\}}
\def\PYZca{\char`\^}
\def\PYZam{\char`\&}
\def\PYZlt{\char`\<}
\def\PYZgt{\char`\>}
\def\PYZsh{\char`\#}
\def\PYZpc{\char`\%}
\def\PYZdl{\char`\$}
\def\PYZhy{\char`\-}
\def\PYZsq{\char`\'}
\def\PYZdq{\char`\"}
\def\PYZti{\char`\~}
% for compatibility with earlier versions
\def\PYZat{@}
\def\PYZlb{[}
\def\PYZrb{]}
\makeatother


    % Exact colors from NB
    \definecolor{incolor}{rgb}{0.0, 0.0, 0.5}
    \definecolor{outcolor}{rgb}{0.545, 0.0, 0.0}



    
    % Prevent overflowing lines due to hard-to-break entities
    \sloppy 
    % Setup hyperref package
    \hypersetup{
      breaklinks=true,  % so long urls are correctly broken across lines
      colorlinks=true,
      urlcolor=urlcolor,
      linkcolor=linkcolor,
      citecolor=citecolor,
      }
    % Slightly bigger margins than the latex defaults
    
    \geometry{verbose,tmargin=1in,bmargin=1in,lmargin=1in,rmargin=1in}
    
    

    \begin{document}
    
    
    \maketitle
    
    

    
    \subsection{Python Built-in functions playground - 75 in
total}\label{python-built-in-functions-playground---75-in-total}

    \begin{Verbatim}[commandchars=\\\{\}]
{\color{incolor}In [{\color{incolor}1}]:} \PY{c+c1}{\PYZsh{} all(iterable) return True if all elements of iterable are True}
        
        \PY{n}{my\PYZus{}list} \PY{o}{=} \PY{p}{[}\PY{l+s+s1}{\PYZsq{}}\PY{l+s+s1}{o}\PY{l+s+s1}{\PYZsq{}}\PY{p}{,} \PY{l+m+mi}{1}\PY{p}{,} \PY{l+m+mi}{2}\PY{p}{,} \PY{l+s+s1}{\PYZsq{}}\PY{l+s+s1}{Text}\PY{l+s+s1}{\PYZsq{}}\PY{p}{,} \PY{k+kc}{True}\PY{p}{,} \PY{l+s+s1}{\PYZsq{}}\PY{l+s+s1}{abc}\PY{l+s+s1}{\PYZsq{}}\PY{p}{]} \PY{c+c1}{\PYZsh{} list elements can be different types}
        \PY{n+nb}{print}\PY{p}{(}\PY{n}{f}\PY{l+s+s2}{\PYZdq{}}\PY{l+s+s2}{My list is: }\PY{l+s+se}{\PYZbs{}n}\PY{l+s+s2}{ }\PY{l+s+si}{\PYZob{}my\PYZus{}list\PYZcb{}}\PY{l+s+s2}{, }\PY{l+s+se}{\PYZbs{}n}\PY{l+s+s2}{ the result of all() is: }\PY{l+s+s2}{\PYZob{}}\PY{l+s+s2}{all(my\PYZus{}list)\PYZcb{}}\PY{l+s+s2}{\PYZdq{}}\PY{p}{)}
        \PY{n+nb}{print}\PY{p}{(}\PY{l+s+s1}{\PYZsq{}}\PY{l+s+s1}{*}\PY{l+s+s1}{\PYZsq{}} \PY{o}{*} \PY{l+m+mi}{15}\PY{p}{)}
        \PY{n}{my\PYZus{}list}\PY{o}{.}\PY{n}{append}\PY{p}{(}\PY{k+kc}{False}\PY{p}{)}
        \PY{n+nb}{print}\PY{p}{(}\PY{n}{f}\PY{l+s+s2}{\PYZdq{}}\PY{l+s+s2}{My list is updated as: }\PY{l+s+se}{\PYZbs{}n}\PY{l+s+s2}{ }\PY{l+s+si}{\PYZob{}my\PYZus{}list\PYZcb{}}\PY{l+s+s2}{, }\PY{l+s+se}{\PYZbs{}n}\PY{l+s+s2}{ the result of all() is now: }\PY{l+s+s2}{\PYZob{}}\PY{l+s+s2}{all(my\PYZus{}list)\PYZcb{}}\PY{l+s+s2}{\PYZdq{}}\PY{p}{)}
\end{Verbatim}


    \begin{Verbatim}[commandchars=\\\{\}]
My list is: 
 ['o', 1, 2, 'Text', True, 'abc'], 
 the result of all() is: True
***************
My list is updated as: 
 ['o', 1, 2, 'Text', True, 'abc', False], 
 the result of all() is now: False

    \end{Verbatim}

    \begin{Verbatim}[commandchars=\\\{\}]
{\color{incolor}In [{\color{incolor}2}]:} \PY{c+c1}{\PYZsh{} any() similar to all()}
        
        \PY{n+nb}{print}\PY{p}{(}\PY{n+nb}{any}\PY{p}{(}\PY{n}{my\PYZus{}list}\PY{p}{)}\PY{p}{)}
\end{Verbatim}


    \begin{Verbatim}[commandchars=\\\{\}]
True

    \end{Verbatim}

    \begin{Verbatim}[commandchars=\\\{\}]
{\color{incolor}In [{\color{incolor}3}]:} \PY{c+c1}{\PYZsh{} ascii() returns the ascii string of a string. 我理解为ascii的参数是想要输出的内容,ascii的结果为对应的输入。相当于给定输出求输入??}
        
        \PY{n+nb}{print}\PY{p}{(}\PY{n}{ascii}\PY{p}{(}\PY{l+s+s1}{\PYZsq{}}\PY{l+s+se}{\PYZbs{}\PYZbs{}}\PY{l+s+s1}{\PYZsq{}}\PY{p}{)} \PY{o}{==} \PY{l+s+s2}{\PYZdq{}}\PY{l+s+s2}{\PYZsq{}}\PY{l+s+se}{\PYZbs{}\PYZbs{}}\PY{l+s+se}{\PYZbs{}\PYZbs{}}\PY{l+s+s2}{\PYZsq{}}\PY{l+s+s2}{\PYZdq{}}\PY{p}{)}
        \PY{n+nb}{print}\PY{p}{(}\PY{n}{ascii}\PY{p}{(}\PY{l+s+s1}{\PYZsq{}}\PY{l+s+se}{\PYZbs{}n}\PY{l+s+s1}{\PYZsq{}}\PY{p}{)} \PY{o}{==} \PY{l+s+s2}{\PYZdq{}}\PY{l+s+s2}{\PYZsq{}}\PY{l+s+se}{\PYZbs{}\PYZbs{}}\PY{l+s+s2}{n}\PY{l+s+s2}{\PYZsq{}}\PY{l+s+s2}{\PYZdq{}}\PY{p}{)}
        \PY{n+nb}{print}\PY{p}{(}\PY{n}{ascii}\PY{p}{(}\PY{l+s+s1}{\PYZsq{}}\PY{l+s+s1}{0b10}\PY{l+s+s1}{\PYZsq{}}\PY{p}{)} \PY{o}{==} \PY{l+s+s2}{\PYZdq{}}\PY{l+s+s2}{\PYZsq{}}\PY{l+s+s2}{2}\PY{l+s+s2}{\PYZsq{}}\PY{l+s+s2}{\PYZdq{}}\PY{p}{)}
\end{Verbatim}


    \begin{Verbatim}[commandchars=\\\{\}]
True
True
False

    \end{Verbatim}

    \begin{Verbatim}[commandchars=\\\{\}]
{\color{incolor}In [{\color{incolor}4}]:} \PY{c+c1}{\PYZsh{} bin() Convert an integer number to a binary string prefixed with “0b”.}
        
        \PY{n+nb}{bin}\PY{p}{(}\PY{l+m+mi}{2}\PY{p}{)}
\end{Verbatim}


\begin{Verbatim}[commandchars=\\\{\}]
{\color{outcolor}Out[{\color{outcolor}4}]:} '0b10'
\end{Verbatim}
            
    \begin{Verbatim}[commandchars=\\\{\}]
{\color{incolor}In [{\color{incolor}5}]:} \PY{n+nb}{format}\PY{p}{(}\PY{l+m+mi}{14}\PY{p}{,} \PY{l+s+s1}{\PYZsq{}}\PY{l+s+s1}{\PYZsh{}b}\PY{l+s+s1}{\PYZsq{}}\PY{p}{)}\PY{p}{,} \PY{n+nb}{format}\PY{p}{(}\PY{l+m+mi}{14}\PY{p}{,} \PY{l+s+s1}{\PYZsq{}}\PY{l+s+s1}{b}\PY{l+s+s1}{\PYZsq{}}\PY{p}{)} \PY{c+c1}{\PYZsh{} 不太懂 要学习一下format函数}
\end{Verbatim}


\begin{Verbatim}[commandchars=\\\{\}]
{\color{outcolor}Out[{\color{outcolor}5}]:} ('0b1110', '1110')
\end{Verbatim}
            
    \begin{Verbatim}[commandchars=\\\{\}]
{\color{incolor}In [{\color{incolor}6}]:} \PY{c+c1}{\PYZsh{} class bool([x]) return a Boolean value, i.e. one of True or False. class部分要学习一下,回头再看这个。}
        
        \PY{c+c1}{\PYZsh{}  x is now a positional\PYZhy{}only parameter??? 这个的含义也不懂。}
\end{Verbatim}


    \begin{Verbatim}[commandchars=\\\{\}]
{\color{incolor}In [{\color{incolor}7}]:} \PY{c+c1}{\PYZsh{} breakpoint(*args, **kws)  debug function.暂时不懂。}
\end{Verbatim}


    \begin{Verbatim}[commandchars=\\\{\}]
{\color{incolor}In [{\color{incolor}8}]:} \PY{c+c1}{\PYZsh{} class bytearray([source[, encoding[, errors]]]) }
\end{Verbatim}


    \subsection{\texorpdfstring{Sequence Types -\/- \textbf{list},
\textbf{tuple},
\textbf{range}}{Sequence Types -\/- list, tuple, range}}\label{sequence-types----list-tuple-range}

    \subsubsection{Common Sequence
Operations}\label{common-sequence-operations}

    The operations in the following table are supported by most sequence
typs, both mutable and immutable.

In this table \textbf{s} and \textbf{t} are sequences of the same type,
\textbf{n}, \textbf{i}, \textbf{j} and \textbf{k} are integers and
\textbf{x} is an arbitrary object that meets any type and value
restrictions imposed by \textbf{s}.

The \textbf{in} and \textbf{not in} have the same priorities as the
comparison operations. The \textbf{+} (concatenation) and \textbf{*}
(repetition) have the same priority as the corresponding numeric
operations.

\begin{longtable}[]{@{}lll@{}}
\toprule
\begin{minipage}[b]{0.14\columnwidth}\raggedright\strut
Operation\strut
\end{minipage} & \begin{minipage}[b]{0.14\columnwidth}\raggedright\strut
Result\strut
\end{minipage} & \begin{minipage}[b]{0.14\columnwidth}\raggedright\strut
Notes\strut
\end{minipage}\tabularnewline
\midrule
\endhead
\begin{minipage}[t]{0.14\columnwidth}\raggedright\strut
x in s\strut
\end{minipage} & \begin{minipage}[t]{0.14\columnwidth}\raggedright\strut
\textbf{True} if an item of s is equal to x, else \textbf{False}\strut
\end{minipage} & \begin{minipage}[t]{0.14\columnwidth}\raggedright\strut
1\strut
\end{minipage}\tabularnewline
\begin{minipage}[t]{0.14\columnwidth}\raggedright\strut
x not in s\strut
\end{minipage} & \begin{minipage}[t]{0.14\columnwidth}\raggedright\strut
\textbf{False} if an item of s is euqal to x, else \textbf{False}\strut
\end{minipage} & \begin{minipage}[t]{0.14\columnwidth}\raggedright\strut
1\strut
\end{minipage}\tabularnewline
\begin{minipage}[t]{0.14\columnwidth}\raggedright\strut
s + t\strut
\end{minipage} & \begin{minipage}[t]{0.14\columnwidth}\raggedright\strut
the concatenation of s and t\strut
\end{minipage} & \begin{minipage}[t]{0.14\columnwidth}\raggedright\strut
6 7\strut
\end{minipage}\tabularnewline
\begin{minipage}[t]{0.14\columnwidth}\raggedright\strut
s * n or n * s\strut
\end{minipage} & \begin{minipage}[t]{0.14\columnwidth}\raggedright\strut
equivalent to adding s to itself n times\strut
\end{minipage} & \begin{minipage}[t]{0.14\columnwidth}\raggedright\strut
2 7\strut
\end{minipage}\tabularnewline
\begin{minipage}[t]{0.14\columnwidth}\raggedright\strut
s{[}i{]}\strut
\end{minipage} & \begin{minipage}[t]{0.14\columnwidth}\raggedright\strut
\emph{i}th item of s, origin 0\strut
\end{minipage} & \begin{minipage}[t]{0.14\columnwidth}\raggedright\strut
3\strut
\end{minipage}\tabularnewline
\begin{minipage}[t]{0.14\columnwidth}\raggedright\strut
s{[}i:j{]}\strut
\end{minipage} & \begin{minipage}[t]{0.14\columnwidth}\raggedright\strut
slice of s from i(inc) to j(exc)\strut
\end{minipage} & \begin{minipage}[t]{0.14\columnwidth}\raggedright\strut
3 4\strut
\end{minipage}\tabularnewline
\begin{minipage}[t]{0.14\columnwidth}\raggedright\strut
s{[}i:j:k{]}\strut
\end{minipage} & \begin{minipage}[t]{0.14\columnwidth}\raggedright\strut
slice of s from i to j with step k\strut
\end{minipage} & \begin{minipage}[t]{0.14\columnwidth}\raggedright\strut
3 5\strut
\end{minipage}\tabularnewline
\begin{minipage}[t]{0.14\columnwidth}\raggedright\strut
len(s)\strut
\end{minipage} & \begin{minipage}[t]{0.14\columnwidth}\raggedright\strut
length of s\strut
\end{minipage} & \begin{minipage}[t]{0.14\columnwidth}\raggedright\strut
\strut
\end{minipage}\tabularnewline
\begin{minipage}[t]{0.14\columnwidth}\raggedright\strut
min(s)\strut
\end{minipage} & \begin{minipage}[t]{0.14\columnwidth}\raggedright\strut
smallest item of s\strut
\end{minipage} & \begin{minipage}[t]{0.14\columnwidth}\raggedright\strut
\strut
\end{minipage}\tabularnewline
\begin{minipage}[t]{0.14\columnwidth}\raggedright\strut
max(s)\strut
\end{minipage} & \begin{minipage}[t]{0.14\columnwidth}\raggedright\strut
largest item of s\strut
\end{minipage} & \begin{minipage}[t]{0.14\columnwidth}\raggedright\strut
\strut
\end{minipage}\tabularnewline
\begin{minipage}[t]{0.14\columnwidth}\raggedright\strut
s.index(x{[}, i{[}, j{]}{]})\strut
\end{minipage} & \begin{minipage}[t]{0.14\columnwidth}\raggedright\strut
index of the first occurrence of x in s (at or after index i and before
index j)\strut
\end{minipage} & \begin{minipage}[t]{0.14\columnwidth}\raggedright\strut
8\strut
\end{minipage}\tabularnewline
\begin{minipage}[t]{0.14\columnwidth}\raggedright\strut
s.count(x)\strut
\end{minipage} & \begin{minipage}[t]{0.14\columnwidth}\raggedright\strut
total number of occurrences of x in s\strut
\end{minipage} & \begin{minipage}[t]{0.14\columnwidth}\raggedright\strut
\strut
\end{minipage}\tabularnewline
\bottomrule
\end{longtable}

\begin{enumerate}
\def\labelenumi{\arabic{enumi}.}
\tightlist
\item
  While the \textbf{in} and \textbf{not in} operations are used only for
  simple containment testing in the general case, some specialised
  sequences (such as \textbf{str}, \textbf{bytes}, \textbf{bytearray})
  also use them for subsequence testing:
\end{enumerate}

    \begin{Verbatim}[commandchars=\\\{\}]
{\color{incolor}In [{\color{incolor}9}]:} \PY{l+s+s2}{\PYZdq{}}\PY{l+s+s2}{gg}\PY{l+s+s2}{\PYZdq{}} \PY{o+ow}{in} \PY{l+s+s2}{\PYZdq{}}\PY{l+s+s2}{eggs}\PY{l+s+s2}{\PYZdq{}}
\end{Verbatim}


\begin{Verbatim}[commandchars=\\\{\}]
{\color{outcolor}Out[{\color{outcolor}9}]:} True
\end{Verbatim}
            
    \begin{enumerate}
\def\labelenumi{\arabic{enumi}.}
\setcounter{enumi}{1}
\tightlist
\item
  Values of \emph{n} less than \textbf{0} are treated as \textbf{0}
  (which yields an empty sequence of the same type as s). Items in the
  sequence \textbf{s} are not copied; they are referenced multiple
  times: \sout{(这个例子不是太理解,后边要继续看append()
  回头看是不是可以理解这里了。)}
\end{enumerate}

    \begin{Verbatim}[commandchars=\\\{\}]
{\color{incolor}In [{\color{incolor}10}]:} \PY{l+s+sd}{\PYZsq{}\PYZsq{}\PYZsq{}Fist of all,we are talking about Sqeuence types here, so the following examle isn\PYZsq{}t part of this topic, it is here only for claritying }
         \PY{l+s+sd}{it is not the same case when multiply non\PYZhy{}list.\PYZsq{}\PYZsq{}\PYZsq{}}
         \PY{c+c1}{\PYZsh{} have to know THIS is wrong to multiply a one\PYZhy{}dimentional list:}
         \PY{n}{lists4} \PY{o}{=} \PY{p}{[}\PY{p}{]} \PY{o}{*} \PY{l+m+mi}{5}
         \PY{n}{lists4}
\end{Verbatim}


\begin{Verbatim}[commandchars=\\\{\}]
{\color{outcolor}Out[{\color{outcolor}10}]:} []
\end{Verbatim}
            
    \begin{Verbatim}[commandchars=\\\{\}]
{\color{incolor}In [{\color{incolor}11}]:} \PY{c+c1}{\PYZsh{} You have to put something in the [] in order to multiply a one\PYZhy{}dimentional list:}
         
         \PY{n}{list5} \PY{o}{=} \PY{p}{[}\PY{l+m+mi}{2}\PY{p}{]} \PY{o}{*} \PY{l+m+mi}{5}
         \PY{n}{list5}
\end{Verbatim}


\begin{Verbatim}[commandchars=\\\{\}]
{\color{outcolor}Out[{\color{outcolor}11}]:} [2, 2, 2, 2, 2]
\end{Verbatim}
            
    \begin{Verbatim}[commandchars=\\\{\}]
{\color{incolor}In [{\color{incolor}12}]:} \PY{c+c1}{\PYZsh{} Now try to change some item: }
         \PY{c+c1}{\PYZsh{} (this example is different from the below, integers are multiplied here, not lists. Thus these items are independent from each other.}
         \PY{c+c1}{\PYZsh{} 不确定这样解释对不对,再看看吧)}
         \PY{n}{list5}\PY{p}{[}\PY{l+m+mi}{0}\PY{p}{]} \PY{o}{=} \PY{l+m+mi}{1}
         \PY{n}{list5}
\end{Verbatim}


\begin{Verbatim}[commandchars=\\\{\}]
{\color{outcolor}Out[{\color{outcolor}12}]:} [1, 2, 2, 2, 2]
\end{Verbatim}
            
    Now we go back to 2.

    \begin{Verbatim}[commandchars=\\\{\}]
{\color{incolor}In [{\color{incolor}13}]:} \PY{n}{lists} \PY{o}{=} \PY{p}{[}\PY{p}{[}\PY{k+kc}{None}\PY{p}{]}\PY{p}{]} \PY{o}{*} \PY{l+m+mi}{5}
         \PY{n}{lists}
         \PY{n}{lists}\PY{p}{[}\PY{l+m+mi}{0}\PY{p}{]}\PY{o}{.}\PY{n}{append}\PY{p}{(}\PY{l+m+mi}{2}\PY{p}{)}
         \PY{n}{lists}
\end{Verbatim}


\begin{Verbatim}[commandchars=\\\{\}]
{\color{outcolor}Out[{\color{outcolor}13}]:} [[None, 2], [None, 2], [None, 2], [None, 2], [None, 2]]
\end{Verbatim}
            
    \begin{Verbatim}[commandchars=\\\{\}]
{\color{incolor}In [{\color{incolor}14}]:} \PY{n}{lists}\PY{p}{[}\PY{l+m+mi}{1}\PY{p}{]}\PY{o}{.}\PY{n}{append}\PY{p}{(}\PY{l+m+mi}{3}\PY{p}{)}
         \PY{n}{lists}
\end{Verbatim}


\begin{Verbatim}[commandchars=\\\{\}]
{\color{outcolor}Out[{\color{outcolor}14}]:} [[None, 2, 3], [None, 2, 3], [None, 2, 3], [None, 2, 3], [None, 2, 3]]
\end{Verbatim}
            
    What happened is that {[}{[}{]}{]} is a one-element list containing an
empty list, so all three elements of {[}{[}{]}{]} * 3 are references to
this single empty list.Modifying any of the elements of lists modifies
this single list. 就是说用 * 新建的list
指向同一个引用,更改一个的内容,其他的一同改变。\href{https://docs.python.org/3/faq/programming.html\#faq-multidimensional-list}{check
this for further explaination}

Use the following method to create a list of different lists this way:

    \begin{Verbatim}[commandchars=\\\{\}]
{\color{incolor}In [{\color{incolor}15}]:} \PY{n}{lists2} \PY{o}{=} \PY{p}{[}\PY{p}{[}\PY{p}{]} \PY{k}{for} \PY{n}{i} \PY{o+ow}{in} \PY{n+nb}{range}\PY{p}{(}\PY{l+m+mi}{3}\PY{p}{)}\PY{p}{]}
         \PY{n}{lists2}\PY{p}{[}\PY{l+m+mi}{0}\PY{p}{]}\PY{o}{.}\PY{n}{append}\PY{p}{(}\PY{l+m+mi}{2}\PY{p}{)}
         \PY{n}{lists2}\PY{p}{[}\PY{l+m+mi}{1}\PY{p}{]}\PY{o}{.}\PY{n}{append}\PY{p}{(}\PY{l+m+mi}{3}\PY{p}{)}
         \PY{n}{lists2}\PY{p}{[}\PY{l+m+mi}{2}\PY{p}{]}\PY{o}{.}\PY{n}{append}\PY{p}{(}\PY{l+m+mi}{5}\PY{p}{)}
         \PY{n}{lists2}
\end{Verbatim}


\begin{Verbatim}[commandchars=\\\{\}]
{\color{outcolor}Out[{\color{outcolor}15}]:} [[2], [3], [5]]
\end{Verbatim}
            
    \begin{Verbatim}[commandchars=\\\{\}]
{\color{incolor}In [{\color{incolor}16}]:} \PY{n}{lists3} \PY{o}{=} \PY{p}{[}\PY{p}{[}\PY{k+kc}{None}\PY{p}{]} \PY{o}{*} \PY{l+m+mi}{2} \PY{k}{for} \PY{n}{i} \PY{o+ow}{in} \PY{n+nb}{range}\PY{p}{(}\PY{l+m+mi}{3}\PY{p}{)}\PY{p}{]}
         \PY{n}{lists3}\PY{p}{[}\PY{l+m+mi}{0}\PY{p}{]}\PY{o}{.}\PY{n}{append}\PY{p}{(}\PY{l+m+mi}{2}\PY{p}{)}
         \PY{n}{lists3}\PY{p}{[}\PY{l+m+mi}{1}\PY{p}{]}\PY{o}{.}\PY{n}{append}\PY{p}{(}\PY{l+m+mi}{3}\PY{p}{)}
         \PY{n}{lists3}\PY{p}{[}\PY{l+m+mi}{2}\PY{p}{]}\PY{o}{.}\PY{n}{append}\PY{p}{(}\PY{l+m+mi}{5}\PY{p}{)}
         \PY{n}{lists3}
\end{Verbatim}


\begin{Verbatim}[commandchars=\\\{\}]
{\color{outcolor}Out[{\color{outcolor}16}]:} [[None, None, 2], [None, None, 3], [None, None, 5]]
\end{Verbatim}
            
    \begin{Verbatim}[commandchars=\\\{\}]
{\color{incolor}In [{\color{incolor}17}]:} \PY{c+c1}{\PYZsh{} items are independent from each other, correct way to initialize a two\PYZhy{}dimentional list: (because [None] * w multiplies a single value, not a list.)}
         \PY{n}{w}\PY{p}{,} \PY{n}{h} \PY{o}{=} \PY{l+m+mi}{2}\PY{p}{,} \PY{l+m+mi}{3}
         \PY{n}{A} \PY{o}{=} \PY{p}{[}\PY{p}{[}\PY{k+kc}{None}\PY{p}{]} \PY{o}{*} \PY{n}{w} \PY{k}{for} \PY{n}{i} \PY{o+ow}{in} \PY{n+nb}{range}\PY{p}{(}\PY{n}{h}\PY{p}{)}\PY{p}{]}
         \PY{n}{A}\PY{p}{[}\PY{l+m+mi}{0}\PY{p}{]}\PY{p}{[}\PY{l+m+mi}{0}\PY{p}{]} \PY{o}{=} \PY{l+m+mi}{1}
         \PY{n}{A}
\end{Verbatim}


\begin{Verbatim}[commandchars=\\\{\}]
{\color{outcolor}Out[{\color{outcolor}17}]:} [[1, None], [None, None], [None, None]]
\end{Verbatim}
            
    \begin{enumerate}
\def\labelenumi{\arabic{enumi}.}
\setcounter{enumi}{2}
\tightlist
\item
  If \emph{i} or \emph{j} is negative, the index is relative to the end
  of sequence s: \textbf{len(s) + i} or \textbf{len(s) + j} is
  substituted. But note that -0 is still 0.
\end{enumerate}

    \begin{enumerate}
\def\labelenumi{\arabic{enumi}.}
\setcounter{enumi}{3}
\tightlist
\item
  \textbf{i} is inclusive, \textbf{j} is exclusive. If \textbf{i} or
  \textbf{j} is greater than \textbf{len(s)}, use \textbf{len(s)}. If
  \textbf{i} is omitted or \textbf{None}, use \textbf{0}. If \textbf{j}
  is omitted or \textbf{None}, use \textbf{len(s)}. If \textbf{i} is
  greater than or equal to \textbf{j}, the slice is empty.
\end{enumerate}

    \begin{enumerate}
\def\labelenumi{\arabic{enumi}.}
\setcounter{enumi}{4}
\tightlist
\item
  s{[}i:j:k{]}, the indices are \textbf{i}, \textbf{i+k}, \textbf{i+2k},
  and so on, stopping when \textbf{j} is reached (but
  exclusive).后边还有解释,不写了,去网上看。。。
\end{enumerate}

    \begin{enumerate}
\def\labelenumi{\arabic{enumi}.}
\setcounter{enumi}{5}
\tightlist
\item
  Concatenating immutable swquences always results in a new object,
  which means runtime cost is much higher!!! 4 ways to save the runtime.
  看网站去。
\end{enumerate}

    \begin{enumerate}
\def\labelenumi{\arabic{enumi}.}
\setcounter{enumi}{6}
\item
\end{enumerate}

    \begin{enumerate}
\def\labelenumi{\arabic{enumi}.}
\setcounter{enumi}{7}
\item
\end{enumerate}

    \subsection{String}\label{string}

    \subsubsection{string basic}\label{string-basic}

    If you don't want characters prefaced by ~to be interpreted as special
characters, you can use raw strings by adding an r before the first
quote:

    \begin{Verbatim}[commandchars=\\\{\}]
{\color{incolor}In [{\color{incolor}18}]:} \PY{n+nb}{print}\PY{p}{(}\PY{l+s+sa}{r}\PY{l+s+s1}{\PYZsq{}}\PY{l+s+s1}{C:}\PY{l+s+s1}{\PYZbs{}}\PY{l+s+s1}{some}\PY{l+s+s1}{\PYZbs{}}\PY{l+s+s1}{name}\PY{l+s+s1}{\PYZsq{}}\PY{p}{)}  \PY{c+c1}{\PYZsh{} note the r before the quote}
\end{Verbatim}


    \begin{Verbatim}[commandchars=\\\{\}]
C:\textbackslash{}some\textbackslash{}name

    \end{Verbatim}

    Two or more string literals (i.e. the ones enclosed between quotes) next
to each other are automatically concatenated.

    \begin{Verbatim}[commandchars=\\\{\}]
{\color{incolor}In [{\color{incolor}19}]:} \PY{l+s+s1}{\PYZsq{}}\PY{l+s+s1}{Py}\PY{l+s+s1}{\PYZsq{}} \PY{l+s+s1}{\PYZsq{}}\PY{l+s+s1}{thon}\PY{l+s+s1}{\PYZsq{}}
\end{Verbatim}


\begin{Verbatim}[commandchars=\\\{\}]
{\color{outcolor}Out[{\color{outcolor}19}]:} 'Python'
\end{Verbatim}
            
    \begin{Verbatim}[commandchars=\\\{\}]
{\color{incolor}In [{\color{incolor}20}]:} \PY{n}{text} \PY{o}{=} \PY{p}{(}\PY{l+s+s1}{\PYZsq{}}\PY{l+s+s1}{Put several strings within parentheses }\PY{l+s+s1}{\PYZsq{}} \PY{l+s+s1}{\PYZsq{}}\PY{l+s+s1}{to have them joined together.}\PY{l+s+s1}{\PYZsq{}}\PY{p}{)}
         \PY{n}{text}
\end{Verbatim}


\begin{Verbatim}[commandchars=\\\{\}]
{\color{outcolor}Out[{\color{outcolor}20}]:} 'Put several strings within parentheses to have them joined together.'
\end{Verbatim}
            
    \begin{Verbatim}[commandchars=\\\{\}]
{\color{incolor}In [{\color{incolor}21}]:} \PY{n+nb}{type}\PY{p}{(}\PY{n}{text}\PY{p}{)}
\end{Verbatim}


\begin{Verbatim}[commandchars=\\\{\}]
{\color{outcolor}Out[{\color{outcolor}21}]:} str
\end{Verbatim}
            
    Python strings cannot be changed --- they are immutable.

    \subsubsection{String method}\label{string-method}

    \begin{Verbatim}[commandchars=\\\{\}]
{\color{incolor}In [{\color{incolor}22}]:} \PY{c+c1}{\PYZsh{} str.strip([char])}
         \PY{l+s+sd}{\PYZsq{}\PYZsq{}\PYZsq{}The OUTERMOST leading and trailing chars argument values are stripped from the string. }
         \PY{l+s+sd}{Unil reaching a string character that is not contained i the set of char, [char].}
         \PY{l+s+sd}{The chars argument is not a prefix or suffix; rather, every single character }
         \PY{l+s+sd}{in it is stripped.}
         \PY{l+s+sd}{\PYZsq{}\PYZsq{}\PYZsq{}}
         \PY{c+c1}{\PYZsh{} example:}
\end{Verbatim}


\begin{Verbatim}[commandchars=\\\{\}]
{\color{outcolor}Out[{\color{outcolor}22}]:} 'The OUTERMOST leading and trailing chars argument values are stripped from the string. \textbackslash{}nUnil reaching a string character that is not contained i the set of char, [char].\textbackslash{}nThe chars argument is not a prefix or suffix; rather, every single character \textbackslash{}nin it is stripped.\textbackslash{}n'
\end{Verbatim}
            
    \begin{Verbatim}[commandchars=\\\{\}]
{\color{incolor}In [{\color{incolor}23}]:} \PY{l+s+s1}{\PYZsq{}}\PY{l+s+s1}{     spacious     }\PY{l+s+s1}{\PYZsq{}}\PY{o}{.}\PY{n}{strip}\PY{p}{(}\PY{p}{)}
\end{Verbatim}


\begin{Verbatim}[commandchars=\\\{\}]
{\color{outcolor}Out[{\color{outcolor}23}]:} 'spacious'
\end{Verbatim}
            
    \begin{Verbatim}[commandchars=\\\{\}]
{\color{incolor}In [{\color{incolor}24}]:} \PY{l+s+s1}{\PYZsq{}}\PY{l+s+s1}{www.example.com}\PY{l+s+s1}{\PYZsq{}}\PY{o}{.}\PY{n}{strip}\PY{p}{(}\PY{l+s+s1}{\PYZsq{}}\PY{l+s+s1}{comwa}\PY{l+s+s1}{\PYZsq{}}\PY{p}{)} \PY{c+c1}{\PYZsh{} pay attention here, a is not stripped, since it is in the inner side.}
\end{Verbatim}


\begin{Verbatim}[commandchars=\\\{\}]
{\color{outcolor}Out[{\color{outcolor}24}]:} '.example.'
\end{Verbatim}
            
    \subsection{Lists}\label{lists}

    Lists are mutable, able to contain different types of item.

    Nest lists:

    \begin{Verbatim}[commandchars=\\\{\}]
{\color{incolor}In [{\color{incolor}25}]:} \PY{n}{a} \PY{o}{=} \PY{p}{[}\PY{l+s+s1}{\PYZsq{}}\PY{l+s+s1}{a}\PY{l+s+s1}{\PYZsq{}}\PY{p}{,} \PY{l+s+s1}{\PYZsq{}}\PY{l+s+s1}{b}\PY{l+s+s1}{\PYZsq{}}\PY{p}{,} \PY{l+s+s1}{\PYZsq{}}\PY{l+s+s1}{c}\PY{l+s+s1}{\PYZsq{}}\PY{p}{]}
         \PY{n}{b} \PY{o}{=} \PY{p}{[}\PY{l+m+mi}{1}\PY{p}{,} \PY{l+m+mi}{2}\PY{p}{,} \PY{l+m+mi}{3}\PY{p}{]}
         \PY{n}{x} \PY{o}{=} \PY{p}{[}\PY{n}{a}\PY{p}{,} \PY{n}{b}\PY{p}{]}
         \PY{n}{x}\PY{p}{[}\PY{l+m+mi}{0}\PY{p}{]}
\end{Verbatim}


\begin{Verbatim}[commandchars=\\\{\}]
{\color{outcolor}Out[{\color{outcolor}25}]:} ['a', 'b', 'c']
\end{Verbatim}
            
    \begin{Verbatim}[commandchars=\\\{\}]
{\color{incolor}In [{\color{incolor}26}]:} \PY{n}{x}\PY{p}{[}\PY{l+m+mi}{0}\PY{p}{]}\PY{p}{[}\PY{l+m+mi}{1}\PY{p}{]}
\end{Verbatim}


\begin{Verbatim}[commandchars=\\\{\}]
{\color{outcolor}Out[{\color{outcolor}26}]:} 'b'
\end{Verbatim}
            
    生成连续数列:

    \begin{Verbatim}[commandchars=\\\{\}]
{\color{incolor}In [{\color{incolor}27}]:} \PY{n+nb}{list}\PY{p}{(}\PY{n+nb}{range}\PY{p}{(}\PY{l+m+mi}{3}\PY{p}{,} \PY{l+m+mi}{9}\PY{p}{)}\PY{p}{)}
\end{Verbatim}


\begin{Verbatim}[commandchars=\\\{\}]
{\color{outcolor}Out[{\color{outcolor}27}]:} [3, 4, 5, 6, 7, 8]
\end{Verbatim}
            
    \subsection{Control Flow Tools}\label{control-flow-tools}

    \subsubsection{if-elif-elif-else}\label{if-elif-elif-else}

    \subsubsection{For Loops}\label{for-loops}

    \begin{itemize}
\tightlist
\item
  Python's for statement iterates over the items of any sequence (a list
  or a string), in the order that they appear in the sequence.
\end{itemize}

    \begin{Verbatim}[commandchars=\\\{\}]
{\color{incolor}In [{\color{incolor}28}]:} \PY{n}{words} \PY{o}{=} \PY{p}{[}\PY{l+s+s1}{\PYZsq{}}\PY{l+s+s1}{cat}\PY{l+s+s1}{\PYZsq{}}\PY{p}{,} \PY{l+s+s1}{\PYZsq{}}\PY{l+s+s1}{window}\PY{l+s+s1}{\PYZsq{}}\PY{p}{,} \PY{l+s+s1}{\PYZsq{}}\PY{l+s+s1}{defenestrate}\PY{l+s+s1}{\PYZsq{}}\PY{p}{]}
         \PY{k}{for} \PY{n}{w} \PY{o+ow}{in} \PY{n}{words}\PY{p}{:}
             \PY{n+nb}{print}\PY{p}{(}\PY{n}{w}\PY{p}{,} \PY{n+nb}{len}\PY{p}{(}\PY{n}{w}\PY{p}{)}\PY{p}{)}
\end{Verbatim}


    \begin{Verbatim}[commandchars=\\\{\}]
cat 3
window 6
defenestrate 12

    \end{Verbatim}

    \begin{itemize}
\tightlist
\item
  If you need to modify the sequence you are iterating over while inside
  the loop (for example to duplicate selected items), it is recommended
  that you first make a copy. Iterating over a sequence does not
  implicitly make a copy. The slice notation makes this especially
  convenient:
\end{itemize}

    \begin{Verbatim}[commandchars=\\\{\}]
{\color{incolor}In [{\color{incolor}29}]:} \PY{k}{for} \PY{n}{w} \PY{o+ow}{in} \PY{n}{words}\PY{p}{[}\PY{p}{:}\PY{p}{]}\PY{p}{:}
             \PY{k}{if} \PY{n+nb}{len}\PY{p}{(}\PY{n}{w}\PY{p}{)} \PY{o}{\PYZgt{}} \PY{l+m+mi}{6}\PY{p}{:}
                 \PY{n}{words}\PY{o}{.}\PY{n}{insert}\PY{p}{(}\PY{l+m+mi}{0}\PY{p}{,} \PY{n}{w}\PY{p}{)} \PY{c+c1}{\PYZsh{} insert w to the 0 position in words}
         \PY{n}{words}
\end{Verbatim}


\begin{Verbatim}[commandchars=\\\{\}]
{\color{outcolor}Out[{\color{outcolor}29}]:} ['defenestrate', 'cat', 'window', 'defenestrate']
\end{Verbatim}
            
    \subsubsection{range() function}\label{range-function}

    \begin{Verbatim}[commandchars=\\\{\}]
{\color{incolor}In [{\color{incolor}30}]:} \PY{k}{for} \PY{n}{i} \PY{o+ow}{in} \PY{n+nb}{range}\PY{p}{(}\PY{l+m+mi}{5}\PY{p}{)}\PY{p}{:}
             \PY{n+nb}{print}\PY{p}{(}\PY{n}{i}\PY{p}{,} \PY{n}{end}\PY{o}{=}\PY{l+s+s1}{\PYZsq{}}\PY{l+s+s1}{, }\PY{l+s+s1}{\PYZsq{}}\PY{p}{)}
\end{Verbatim}


    \begin{Verbatim}[commandchars=\\\{\}]
0, 1, 2, 3, 4, 
    \end{Verbatim}

    \begin{Verbatim}[commandchars=\\\{\}]
{\color{incolor}In [{\color{incolor}31}]:} \PY{k}{for} \PY{n}{i} \PY{o+ow}{in} \PY{n+nb}{range}\PY{p}{(}\PY{l+m+mi}{3}\PY{p}{,} \PY{l+m+mi}{5}\PY{p}{)}\PY{p}{:}
             \PY{n+nb}{print}\PY{p}{(}\PY{n}{i}\PY{p}{,} \PY{n}{end}\PY{o}{=}\PY{l+s+s1}{\PYZsq{}}\PY{l+s+s1}{, }\PY{l+s+s1}{\PYZsq{}}\PY{p}{)}
\end{Verbatim}


    \begin{Verbatim}[commandchars=\\\{\}]
3, 4, 
    \end{Verbatim}

    \begin{Verbatim}[commandchars=\\\{\}]
{\color{incolor}In [{\color{incolor}32}]:} \PY{k}{for} \PY{n}{i} \PY{o+ow}{in} \PY{n+nb}{range}\PY{p}{(}\PY{l+m+mi}{1}\PY{p}{,} \PY{l+m+mi}{20}\PY{p}{,} \PY{l+m+mi}{3}\PY{p}{)}\PY{p}{:}
             \PY{n+nb}{print}\PY{p}{(}\PY{n}{i}\PY{p}{,} \PY{n}{end}\PY{o}{=}\PY{l+s+s1}{\PYZsq{}}\PY{l+s+s1}{, }\PY{l+s+s1}{\PYZsq{}}\PY{p}{)}
\end{Verbatim}


    \begin{Verbatim}[commandchars=\\\{\}]
1, 4, 7, 10, 13, 16, 19, 
    \end{Verbatim}

    \subsubsection{break and continue Statements, and else Clauses on
Loops}\label{break-and-continue-statements-and-else-clauses-on-loops}

    \begin{enumerate}
\def\labelenumi{\arabic{enumi}.}
\tightlist
\item
  break, breaks out of the innermost enclosing \emph{for} or
  \emph{while} loop
\end{enumerate}

    \begin{enumerate}
\def\labelenumi{\arabic{enumi}.}
\setcounter{enumi}{1}
\tightlist
\item
  else clause: is executed when the loop terminates through exhaustion
  of the list (with \emph{for} loop) or when the condition becomes false
  (with \emph{while} loop), but not when the loop is terminated by a
  \emph{break} statement,下面举一个找质数(prime number)的例子:
\end{enumerate}

    \begin{Verbatim}[commandchars=\\\{\}]
{\color{incolor}In [{\color{incolor}33}]:} \PY{n}{m} \PY{o}{=} \PY{l+m+mi}{10} \PY{c+c1}{\PYZsh{} searching for all the prime numbers from zero to m.}
         \PY{k}{for} \PY{n}{n} \PY{o+ow}{in} \PY{n+nb}{range}\PY{p}{(}\PY{l+m+mi}{2}\PY{p}{,} \PY{n}{m}\PY{p}{)}\PY{p}{:}
             \PY{k}{for} \PY{n}{x} \PY{o+ow}{in} \PY{n+nb}{range}\PY{p}{(}\PY{l+m+mi}{2}\PY{p}{,} \PY{n}{n}\PY{p}{)}\PY{p}{:}
                 \PY{k}{if} \PY{n}{n} \PY{o}{\PYZpc{}} \PY{n}{x} \PY{o}{==} \PY{l+m+mi}{0}\PY{p}{:}
                     \PY{n+nb}{print}\PY{p}{(}\PY{n}{n}\PY{p}{,} \PY{l+s+s1}{\PYZsq{}}\PY{l+s+s1}{equals}\PY{l+s+s1}{\PYZsq{}}\PY{p}{,} \PY{n}{x}\PY{p}{,} \PY{l+s+s1}{\PYZsq{}}\PY{l+s+s1}{*}\PY{l+s+s1}{\PYZsq{}}\PY{p}{,} \PY{n}{n}\PY{o}{/}\PY{o}{/}\PY{n}{x}\PY{p}{)}
                     \PY{k}{break}
             \PY{k}{else}\PY{p}{:}
                 \PY{c+c1}{\PYZsh{} loop fell through without finding a factor}
                 \PY{n+nb}{print}\PY{p}{(}\PY{n}{n}\PY{p}{,} \PY{l+s+s1}{\PYZsq{}}\PY{l+s+s1}{is a prime number.}\PY{l+s+s1}{\PYZsq{}}\PY{p}{)}
         \PY{l+s+sd}{\PYZsq{}\PYZsq{}\PYZsq{}\PYZbs{}}
         \PY{l+s+sd}{In the above example, line3 and line7 are a pair }
         \PY{l+s+sd}{(if for loop start from line3 finishes without a break,}
         \PY{l+s+sd}{the else clause will be executed).\PYZbs{}}
         \PY{l+s+sd}{\PYZsq{}\PYZsq{}\PYZsq{}}
\end{Verbatim}


    \begin{Verbatim}[commandchars=\\\{\}]
2 is a prime number.
3 is a prime number.
4 equals 2 * 2
5 is a prime number.
6 equals 2 * 3
7 is a prime number.
8 equals 2 * 4
9 equals 3 * 3

    \end{Verbatim}

\begin{Verbatim}[commandchars=\\\{\}]
{\color{outcolor}Out[{\color{outcolor}33}]:} 'In the above example, line3 and line7 are a pair \textbackslash{}n(if for loop start from line3 finishes without a break,\textbackslash{}nthe else clause will be executed).'
\end{Verbatim}
            
    \begin{enumerate}
\def\labelenumi{\arabic{enumi}.}
\setcounter{enumi}{2}
\tightlist
\item
  The \emph{continue} statement, contnues with the next iteration of the
  loop.
\end{enumerate}

    \begin{Verbatim}[commandchars=\\\{\}]
{\color{incolor}In [{\color{incolor}34}]:} \PY{n}{m} \PY{o}{=} \PY{l+m+mi}{10}
         \PY{k}{for} \PY{n}{num} \PY{o+ow}{in} \PY{n+nb}{range}\PY{p}{(}\PY{l+m+mi}{2}\PY{p}{,} \PY{n}{m}\PY{p}{)}\PY{p}{:}
             \PY{k}{if} \PY{n}{num} \PY{o}{\PYZpc{}} \PY{l+m+mi}{2} \PY{o}{==} \PY{l+m+mi}{0}\PY{p}{:}
                 \PY{n+nb}{print}\PY{p}{(}\PY{l+s+s2}{\PYZdq{}}\PY{l+s+s2}{Found an even number}\PY{l+s+s2}{\PYZdq{}}\PY{p}{,} \PY{n}{num}\PY{p}{)}
                 \PY{k}{continue}
             \PY{n+nb}{print}\PY{p}{(}\PY{l+s+s2}{\PYZdq{}}\PY{l+s+s2}{Found an odd number}\PY{l+s+s2}{\PYZdq{}}\PY{p}{,} \PY{n}{num}\PY{p}{)}
                       
\end{Verbatim}


    \begin{Verbatim}[commandchars=\\\{\}]
Found an even number 2
Found an odd number 3
Found an even number 4
Found an odd number 5
Found an even number 6
Found an odd number 7
Found an even number 8
Found an odd number 9

    \end{Verbatim}

    \subsubsection{pass Statements}\label{pass-statements}

    pass does nothing. It can be used when a statement is required
syntactically but the program requires no action:

    \begin{Verbatim}[commandchars=\\\{\}]
{\color{incolor}In [{\color{incolor} }]:} \PY{k}{while} \PY{k+kc}{True}\PY{p}{:}
            \PY{k}{pass} \PY{c+c1}{\PYZsh{} Busy\PYZhy{}wait for keyboard interrupt}
\end{Verbatim}


    \begin{Verbatim}[commandchars=\\\{\}]
{\color{incolor}In [{\color{incolor} }]:} \PY{k}{class} \PY{n+nc}{MyEnptyClass}\PY{p}{:}
            \PY{k}{pass}
\end{Verbatim}


    \begin{Verbatim}[commandchars=\\\{\}]
{\color{incolor}In [{\color{incolor} }]:} \PY{k}{def} \PY{n+nf}{initlog}\PY{p}{(}\PY{o}{*}\PY{n}{args}\PY{p}{)}\PY{p}{:}
            \PY{k}{pass} \PY{c+c1}{\PYZsh{} Remember to implement this!}
\end{Verbatim}


    \subsubsection{Looping Techniques}\label{looping-techniques}

    \begin{enumerate}
\def\labelenumi{\arabic{enumi}.}
\tightlist
\item
  When looping through dictionaries, the key and corresponding value can
  be retrieved at the same time using the items() method.
\end{enumerate}

    \begin{Verbatim}[commandchars=\\\{\}]
{\color{incolor}In [{\color{incolor} }]:} \PY{n}{knights} \PY{o}{=} \PY{p}{\PYZob{}}\PY{l+s+s1}{\PYZsq{}}\PY{l+s+s1}{gallahad}\PY{l+s+s1}{\PYZsq{}}\PY{p}{:} \PY{l+s+s1}{\PYZsq{}}\PY{l+s+s1}{the pure}\PY{l+s+s1}{\PYZsq{}}\PY{p}{,} \PY{l+s+s1}{\PYZsq{}}\PY{l+s+s1}{robin}\PY{l+s+s1}{\PYZsq{}}\PY{p}{:} \PY{l+s+s1}{\PYZsq{}}\PY{l+s+s1}{the brave}\PY{l+s+s1}{\PYZsq{}}\PY{p}{\PYZcb{}}
        \PY{k}{for} \PY{n}{k}\PY{p}{,} \PY{n}{v} \PY{o+ow}{in} \PY{n}{knights}\PY{o}{.}\PY{n}{items}\PY{p}{(}\PY{p}{)}\PY{p}{:}
            \PY{n+nb}{print}\PY{p}{(}\PY{n}{f}\PY{l+s+s1}{\PYZsq{}}\PY{l+s+si}{\PYZob{}k\PYZcb{}}\PY{l+s+s1}{: }\PY{l+s+si}{\PYZob{}v\PYZcb{}}\PY{l+s+s1}{\PYZsq{}}\PY{p}{)}
\end{Verbatim}


    \subsubsection{Defining Functions}\label{defining-functions}

    \paragraph{Function return:}\label{function-return}

\begin{itemize}
\tightlist
\item
  return without an expression argument returns \textbf{None}
\item
  Falling off the end of a function also returns \textbf{None}
\end{itemize}

    \subsubsection{More on Defining
Functions}\label{more-on-defining-functions}

    \paragraph{Default Argument Values:}\label{default-argument-values}

\begin{itemize}
\tightlist
\item
  Specifying a default value for an argument, makes the argument
  \textbf{optional} when calling the function.
\item
  The default value is only evaluated once, when the function is
  defined. But, if the optional argument is a mutable object, and the
  function is called multiple times without specifying this argument,
  this value could be accumulated in the following example:
\end{itemize}

    \begin{Verbatim}[commandchars=\\\{\}]
{\color{incolor}In [{\color{incolor} }]:} \PY{k}{def} \PY{n+nf}{f}\PY{p}{(}\PY{n}{a}\PY{p}{,} \PY{n}{L}\PY{o}{=}\PY{p}{[}\PY{p}{]}\PY{p}{)}\PY{p}{:}
            \PY{n}{L}\PY{o}{.}\PY{n}{append}\PY{p}{(}\PY{n}{a}\PY{p}{)}
            \PY{k}{return} \PY{n}{L}
        
        \PY{n+nb}{print}\PY{p}{(}\PY{n}{f}\PY{p}{(}\PY{l+m+mi}{1}\PY{p}{)}\PY{p}{)}
        \PY{n+nb}{print}\PY{p}{(}\PY{n}{f}\PY{p}{(}\PY{l+m+mi}{2}\PY{p}{)}\PY{p}{)}
        \PY{n+nb}{print}\PY{p}{(}\PY{n}{f}\PY{p}{(}\PY{l+m+mi}{3}\PY{p}{)}\PY{p}{)}
\end{Verbatim}


    \paragraph{Keyword Arguments}\label{keyword-arguments}

    \begin{enumerate}
\def\labelenumi{\arabic{enumi}.}
\tightlist
\item
  Passing arguments to a function, can use \textbf{positional} or
  \textbf{keyword} arguments. More detailed are in
  \href{docs.python.org/3/tutorial/controlflow.html}{Python Docs}
\end{enumerate}

    \begin{enumerate}
\def\labelenumi{\arabic{enumi}.}
\setcounter{enumi}{1}
\tightlist
\item
  Two more options of passing arguments (these are \textbf{arbitrary
  number} of arguments):

  \begin{itemize}
  \tightlist
  \item
    *name: It recieves a tuple containing the \textbf{positional}
    arguments beyond the formal parameter list.
  \item
    **name: It recevices a dictionary containing all \textbf{keyword}
    arguments except for those corresponding to a formal parameter.
  \end{itemize}
\end{enumerate}

    \begin{Verbatim}[commandchars=\\\{\}]
{\color{incolor}In [{\color{incolor} }]:} \PY{k}{def} \PY{n+nf}{cheeseshop}\PY{p}{(}\PY{n}{kind}\PY{p}{,} \PY{o}{*}\PY{n}{arguments}\PY{p}{,} \PY{o}{*}\PY{o}{*}\PY{n}{keywords}\PY{p}{)}\PY{p}{:}
            \PY{n+nb}{print}\PY{p}{(}\PY{l+s+s2}{\PYZdq{}}\PY{l+s+s2}{\PYZhy{}\PYZhy{} Do you have any}\PY{l+s+s2}{\PYZdq{}}\PY{p}{,} \PY{n}{kind}\PY{p}{,} \PY{l+s+s2}{\PYZdq{}}\PY{l+s+s2}{?}\PY{l+s+s2}{\PYZdq{}}\PY{p}{)}
            \PY{n+nb}{print}\PY{p}{(}\PY{n}{f}\PY{l+s+s2}{\PYZdq{}}\PY{l+s+s2}{\PYZhy{}\PYZhy{} I}\PY{l+s+s2}{\PYZsq{}}\PY{l+s+s2}{m sorry, we}\PY{l+s+s2}{\PYZsq{}}\PY{l+s+s2}{re all out of }\PY{l+s+si}{\PYZob{}kind\PYZcb{}}\PY{l+s+s2}{.}\PY{l+s+s2}{\PYZdq{}}\PY{p}{)}
            \PY{k}{for} \PY{n}{arg} \PY{o+ow}{in} \PY{n}{arguments}\PY{p}{:}
                \PY{n+nb}{print}\PY{p}{(}\PY{n}{arg}\PY{p}{)}
            \PY{n+nb}{print}\PY{p}{(}\PY{l+s+s1}{\PYZsq{}}\PY{l+s+s1}{*}\PY{l+s+s1}{\PYZsq{}} \PY{o}{*} \PY{l+m+mi}{40}\PY{p}{)}
            \PY{k}{for} \PY{n}{key} \PY{o+ow}{in} \PY{n}{keywords}\PY{p}{:}
                \PY{n+nb}{print}\PY{p}{(}\PY{n}{key}\PY{p}{,} \PY{l+s+s2}{\PYZdq{}}\PY{l+s+s2}{:}\PY{l+s+s2}{\PYZdq{}}\PY{p}{,} \PY{n}{keywords}\PY{p}{[}\PY{n}{key}\PY{p}{]}\PY{p}{)}
\end{Verbatim}


    The above could be called like this:

    \begin{Verbatim}[commandchars=\\\{\}]
{\color{incolor}In [{\color{incolor} }]:} \PY{n}{cheeseshop}\PY{p}{(}\PY{l+s+s2}{\PYZdq{}}\PY{l+s+s2}{Limburger}\PY{l+s+s2}{\PYZdq{}}\PY{p}{,} \PY{l+s+s2}{\PYZdq{}}\PY{l+s+s2}{It}\PY{l+s+s2}{\PYZsq{}}\PY{l+s+s2}{s very runny, sir.}\PY{l+s+s2}{\PYZdq{}}\PY{p}{,}
                  \PY{l+s+s2}{\PYZdq{}}\PY{l+s+s2}{It}\PY{l+s+s2}{\PYZsq{}}\PY{l+s+s2}{s really very, VERY runny, sir.}\PY{l+s+s2}{\PYZdq{}}\PY{p}{,}
                  \PY{n}{shopkeeper}\PY{o}{=}\PY{l+s+s2}{\PYZdq{}}\PY{l+s+s2}{Michael Palin}\PY{l+s+s2}{\PYZdq{}}\PY{p}{,}
                  \PY{n}{client}\PY{o}{=}\PY{l+s+s2}{\PYZdq{}}\PY{l+s+s2}{John Cleese}\PY{l+s+s2}{\PYZdq{}}\PY{p}{,}
                  \PY{n}{sketch}\PY{o}{=}\PY{l+s+s2}{\PYZdq{}}\PY{l+s+s2}{Cheese Shop Sketch}\PY{l+s+s2}{\PYZdq{}}\PY{p}{)}
\end{Verbatim}


    \paragraph{Unpacking Argument Lists}\label{unpacking-argument-lists}

    \begin{Verbatim}[commandchars=\\\{\}]
{\color{incolor}In [{\color{incolor} }]:} \PY{n+nb}{list}\PY{p}{(}\PY{n+nb}{range}\PY{p}{(}\PY{l+m+mi}{3}\PY{p}{,} \PY{l+m+mi}{6}\PY{p}{)}\PY{p}{)}    \PY{c+c1}{\PYZsh{} Normal call with separate arguments}
\end{Verbatim}


    \begin{Verbatim}[commandchars=\\\{\}]
{\color{incolor}In [{\color{incolor} }]:} \PY{n}{args} \PY{o}{=} \PY{p}{[}\PY{l+m+mi}{3}\PY{p}{,} \PY{l+m+mi}{6}\PY{p}{]}
        \PY{n+nb}{list}\PY{p}{(}\PY{n+nb}{range}\PY{p}{(}\PY{o}{*}\PY{n}{args}\PY{p}{)}\PY{p}{)} \PY{c+c1}{\PYZsh{} call with arguments unpacked from a list}
\end{Verbatim}


    \paragraph{Lambda Expressions}\label{lambda-expressions}

    Small anonymous functions can be created with the \textbf{lambda}
keyword.这个例子仔细看,make\_incrementor是个关于n的函数,f是个关于x的函数。

    def make\_incrementor(n): return lambda x: x + n

f = make\_incrementor(42) f(1)

    Another use is to pass a small function as an argument.

    \begin{Verbatim}[commandchars=\\\{\}]
{\color{incolor}In [{\color{incolor} }]:} \PY{n}{pairs} \PY{o}{=} \PY{p}{[}\PY{p}{(}\PY{l+m+mi}{1}\PY{p}{,} \PY{l+s+s1}{\PYZsq{}}\PY{l+s+s1}{one}\PY{l+s+s1}{\PYZsq{}}\PY{p}{)}\PY{p}{,} \PY{p}{(}\PY{l+m+mi}{2}\PY{p}{,} \PY{l+s+s1}{\PYZsq{}}\PY{l+s+s1}{two}\PY{l+s+s1}{\PYZsq{}}\PY{p}{)}\PY{p}{,} \PY{p}{(}\PY{l+m+mi}{3}\PY{p}{,} \PY{l+s+s1}{\PYZsq{}}\PY{l+s+s1}{three}\PY{l+s+s1}{\PYZsq{}}\PY{p}{)}\PY{p}{,} \PY{p}{(}\PY{l+m+mi}{4}\PY{p}{,} \PY{l+s+s1}{\PYZsq{}}\PY{l+s+s1}{four}\PY{l+s+s1}{\PYZsq{}}\PY{p}{)}\PY{p}{]}
        \PY{n}{pairs}\PY{o}{.}\PY{n}{sort}\PY{p}{(}\PY{n}{key}\PY{o}{=}\PY{k}{lambda} \PY{n}{pair}\PY{p}{:} \PY{n}{pair}\PY{p}{[}\PY{l+m+mi}{1}\PY{p}{]}\PY{p}{)}    \PY{c+c1}{\PYZsh{}按里层list第二个元素排序 也就是 按照one two three four的首字母排序 下边list.sort()有进一步解释}
        \PY{n}{pairs}
\end{Verbatim}


    \paragraph{Function Annotations}\label{function-annotations}

    暂时用不到,用来说明自定义函数的参数类型,方便阅读代码的。

    \subsubsection{Coding Style}\label{coding-style}

    \begin{enumerate}
\def\labelenumi{\arabic{enumi}.}
\tightlist
\item
  Use 4-space indentation, and no \textbf{tabs}.
\end{enumerate}

    \begin{enumerate}
\def\labelenumi{\arabic{enumi}.}
\setcounter{enumi}{1}
\tightlist
\item
  Wrap lines so that they don't exceed 79 characters.
\end{enumerate}

    \begin{enumerate}
\def\labelenumi{\arabic{enumi}.}
\setcounter{enumi}{2}
\tightlist
\item
  Use blank lines to serarate functions and classes, and larger blocks
  of code inside functions.
\end{enumerate}

    \begin{enumerate}
\def\labelenumi{\arabic{enumi}.}
\setcounter{enumi}{3}
\tightlist
\item
  When possible, put comments on a line of their own.
\end{enumerate}

    \begin{enumerate}
\def\labelenumi{\arabic{enumi}.}
\setcounter{enumi}{4}
\tightlist
\item
  Use docstrings. ?????
\end{enumerate}

    \begin{enumerate}
\def\labelenumi{\arabic{enumi}.}
\setcounter{enumi}{5}
\item
  Use spaces around operators and after commas, but not directly inside
  bracketing constructs: (此处不太懂具体说的什么意思。)

  \begin{itemize}
  \tightlist
  \item
    a = f(1, 3) + g(3, 4)
  \end{itemize}
\end{enumerate}

    \begin{enumerate}
\def\labelenumi{\arabic{enumi}.}
\setcounter{enumi}{6}
\tightlist
\item
  Name your classes and functions consistently; the convention is to use
  \textbf{CamelCase} (每个单词首字母大写,单词间不空格) for classes and
  \textbf{lower\_case\_with\_underscores}(全部小写,单词间用下划线分隔)
  for functions and methods.
\end{enumerate}

    \begin{enumerate}
\def\labelenumi{\arabic{enumi}.}
\setcounter{enumi}{7}
\tightlist
\item
  Encodings: Python's default is UTF-8, or even plain ASCII works best
  in any case.
\end{enumerate}

    \subsection{Data Structures}\label{data-structures}

    \subsubsection{methods of list type}\label{methods-of-list-type}

    \begin{enumerate}
\def\labelenumi{\arabic{enumi}.}
\tightlist
\item
  list.append(x)
\end{enumerate}

Add an item to the end of the list. Equivalent to a{[}len(a):{]} =
{[}x{]} \# 此处x可以是含有多项的list

    \begin{enumerate}
\def\labelenumi{\arabic{enumi}.}
\setcounter{enumi}{1}
\tightlist
\item
  list.extend(iterable)
\end{enumerate}

Add an item to the end of the list. Equivalent to a{[}len(a):{]} =
{[}iterable{]}

    以上append 跟 extend 的区别在于
append的参数作为一个整体被加到list的最后;extend的参数必须是iterable的(看下边例子
出错的这个),拆分开分别加进list。

    \begin{Verbatim}[commandchars=\\\{\}]
{\color{incolor}In [{\color{incolor}2}]:} \PY{n}{list\PYZus{}a} \PY{o}{=} \PY{p}{[}\PY{l+m+mi}{1}\PY{p}{,} \PY{l+m+mi}{2}\PY{p}{,} \PY{l+m+mi}{3}\PY{p}{,} \PY{l+m+mi}{4}\PY{p}{,} \PY{l+m+mi}{5}\PY{p}{]}
        \PY{n}{list\PYZus{}a}\PY{o}{.}\PY{n}{append}\PY{p}{(}\PY{l+m+mi}{6}\PY{p}{)}
        \PY{n}{list\PYZus{}a}
\end{Verbatim}


\begin{Verbatim}[commandchars=\\\{\}]
{\color{outcolor}Out[{\color{outcolor}2}]:} [1, 2, 3, 4, 5, 6]
\end{Verbatim}
            
    \begin{Verbatim}[commandchars=\\\{\}]
{\color{incolor}In [{\color{incolor}5}]:} \PY{n}{list\PYZus{}c} \PY{o}{=} \PY{p}{[}\PY{l+m+mi}{1}\PY{p}{,} \PY{l+m+mi}{2}\PY{p}{,} \PY{l+m+mi}{3}\PY{p}{,} \PY{l+m+mi}{4}\PY{p}{,} \PY{l+m+mi}{5}\PY{p}{]}
        \PY{n}{list\PYZus{}c}\PY{o}{.}\PY{n}{append}\PY{p}{(}\PY{p}{[}\PY{l+m+mi}{6}\PY{p}{,} \PY{l+m+mi}{7}\PY{p}{,} \PY{l+m+mi}{8}\PY{p}{]}\PY{p}{)}
        \PY{n}{list\PYZus{}c}
\end{Verbatim}


\begin{Verbatim}[commandchars=\\\{\}]
{\color{outcolor}Out[{\color{outcolor}5}]:} [1, 2, 3, 4, 5, [6, 7, 8]]
\end{Verbatim}
            
    \begin{Verbatim}[commandchars=\\\{\}]
{\color{incolor}In [{\color{incolor}3}]:} \PY{n}{list\PYZus{}b} \PY{o}{=} \PY{p}{[}\PY{l+m+mi}{1}\PY{p}{,} \PY{l+m+mi}{2}\PY{p}{,} \PY{l+m+mi}{3}\PY{p}{,} \PY{l+m+mi}{4}\PY{p}{,} \PY{l+m+mi}{5}\PY{p}{]}
        \PY{n}{list\PYZus{}b}\PY{o}{.}\PY{n}{extend}\PY{p}{(}\PY{p}{[}\PY{l+m+mi}{6}\PY{p}{,} \PY{l+m+mi}{7}\PY{p}{,} \PY{l+m+mi}{8}\PY{p}{]}\PY{p}{)}
        \PY{n}{list\PYZus{}b}
\end{Verbatim}


\begin{Verbatim}[commandchars=\\\{\}]
{\color{outcolor}Out[{\color{outcolor}3}]:} [1, 2, 3, 4, 5, 6, 7, 8]
\end{Verbatim}
            
    \begin{Verbatim}[commandchars=\\\{\}]
{\color{incolor}In [{\color{incolor}6}]:} \PY{n}{list\PYZus{}d} \PY{o}{=} \PY{p}{[}\PY{l+m+mi}{1}\PY{p}{,} \PY{l+m+mi}{2}\PY{p}{,} \PY{l+m+mi}{3}\PY{p}{,} \PY{l+m+mi}{4}\PY{p}{,} \PY{l+m+mi}{5}\PY{p}{]}
        \PY{n}{list\PYZus{}d}\PY{o}{.}\PY{n}{extend}\PY{p}{(}\PY{l+m+mi}{6}\PY{p}{)}
        \PY{n}{list\PYZus{}d}
\end{Verbatim}


    \begin{Verbatim}[commandchars=\\\{\}]

        ---------------------------------------------------------------------------

        TypeError                                 Traceback (most recent call last)

        <ipython-input-6-8387a7ab0753> in <module>()
          1 list\_d = [1, 2, 3, 4, 5]
    ----> 2 list\_d.extend(6)
          3 list\_d
    

        TypeError: 'int' object is not iterable

    \end{Verbatim}

    \begin{enumerate}
\def\labelenumi{\arabic{enumi}.}
\setcounter{enumi}{2}
\tightlist
\item
  list.insert(i, x)
\end{enumerate}

Insert an item at a given position. * i is the index of the element
before which to insert, so \textbf{a.insert(0, x)} inserts at the front
of the list. * \textbf{a.insert(len(a), x)} is equivalent to
\textbf{a.append(x)}

    \begin{enumerate}
\def\labelenumi{\arabic{enumi}.}
\setcounter{enumi}{3}
\item
  list.remove(x)

  \begin{itemize}
  \tightlist
  \item
    Remove the first item from the list whose value is equal to x.
  \item
    It raises a \textbf{ValueError} if there is no such item.
  \end{itemize}
\end{enumerate}

    \begin{enumerate}
\def\labelenumi{\arabic{enumi}.}
\setcounter{enumi}{4}
\tightlist
\item
  list.pop({[}i{]}) \# {[}{]}代表可选参数,参数默认值为-1.
\end{enumerate}

Remove the item at the given position in the list, and return it. If no
index is specified, it removes and returns the last item in the list.

    \begin{enumerate}
\def\labelenumi{\arabic{enumi}.}
\setcounter{enumi}{5}
\tightlist
\item
  list.clear()
\end{enumerate}

Remove all items from the list. Equivalent to \textbf{del list{[}:{]}}

    \begin{enumerate}
\def\labelenumi{\arabic{enumi}.}
\setcounter{enumi}{6}
\item
  list.index(x{[}, start{[}, end{]}{]})

  \begin{itemize}
  \item
    Return zero-based index in the list of the first item whose value is
    equal to x. Raises a \textbf{ValueError} if there is no such item.
  \item
    The optional arguments are the slice notation. The returned index is
    still computed relative to the beginning of the full list.
  \end{itemize}
\end{enumerate}

    \begin{enumerate}
\def\labelenumi{\arabic{enumi}.}
\setcounter{enumi}{7}
\tightlist
\item
  list.count(x)
\end{enumerate}

Return the number of times x appears in the list.

    \begin{enumerate}
\def\labelenumi{\arabic{enumi}.}
\setcounter{enumi}{8}
\tightlist
\item
  list.sort(key=None, reverse=False)
\end{enumerate}

Sort the items of the list in place. key specifies a function of one
argument that is used to extract a comparison key from each list element
(key 必须是个function(lambda function)???)

    \begin{Verbatim}[commandchars=\\\{\}]
{\color{incolor}In [{\color{incolor}7}]:} \PY{n}{pairs} \PY{o}{=} \PY{p}{[}\PY{p}{(}\PY{l+m+mi}{1}\PY{p}{,} \PY{l+s+s1}{\PYZsq{}}\PY{l+s+s1}{one}\PY{l+s+s1}{\PYZsq{}}\PY{p}{)}\PY{p}{,} \PY{p}{(}\PY{l+m+mi}{2}\PY{p}{,} \PY{l+s+s1}{\PYZsq{}}\PY{l+s+s1}{two}\PY{l+s+s1}{\PYZsq{}}\PY{p}{)}\PY{p}{,} \PY{p}{(}\PY{l+m+mi}{3}\PY{p}{,} \PY{l+s+s1}{\PYZsq{}}\PY{l+s+s1}{three}\PY{l+s+s1}{\PYZsq{}}\PY{p}{)}\PY{p}{,} \PY{p}{(}\PY{l+m+mi}{4}\PY{p}{,} \PY{l+s+s1}{\PYZsq{}}\PY{l+s+s1}{four}\PY{l+s+s1}{\PYZsq{}}\PY{p}{)}\PY{p}{]}
        \PY{n}{pairs}\PY{o}{.}\PY{n}{sort}\PY{p}{(}\PY{n}{key}\PY{o}{=}\PY{k}{lambda} \PY{n}{pair}\PY{p}{:} \PY{n}{pair}\PY{p}{[}\PY{l+m+mi}{1}\PY{p}{]}\PY{p}{)}    \PY{c+c1}{\PYZsh{}按里层list第二个元素排序 也就是 按照one two three four的首字母排序 下边list.sort()有进一步解释}
        \PY{n}{pairs}
\end{Verbatim}


\begin{Verbatim}[commandchars=\\\{\}]
{\color{outcolor}Out[{\color{outcolor}7}]:} [(4, 'four'), (1, 'one'), (3, 'three'), (2, 'two')]
\end{Verbatim}
            
    \begin{Verbatim}[commandchars=\\\{\}]
{\color{incolor}In [{\color{incolor}9}]:} \PY{n}{pairs} \PY{o}{=} \PY{p}{[}\PY{p}{(}\PY{l+m+mi}{1}\PY{p}{,} \PY{l+s+s1}{\PYZsq{}}\PY{l+s+s1}{one}\PY{l+s+s1}{\PYZsq{}}\PY{p}{)}\PY{p}{,} \PY{p}{(}\PY{l+m+mi}{2}\PY{p}{,} \PY{l+s+s1}{\PYZsq{}}\PY{l+s+s1}{two}\PY{l+s+s1}{\PYZsq{}}\PY{p}{)}\PY{p}{,} \PY{p}{(}\PY{l+m+mi}{3}\PY{p}{,} \PY{l+s+s1}{\PYZsq{}}\PY{l+s+s1}{three}\PY{l+s+s1}{\PYZsq{}}\PY{p}{)}\PY{p}{,} \PY{p}{(}\PY{l+m+mi}{4}\PY{p}{,} \PY{l+s+s1}{\PYZsq{}}\PY{l+s+s1}{four}\PY{l+s+s1}{\PYZsq{}}\PY{p}{)}\PY{p}{]}
        \PY{n}{pairs}\PY{o}{.}\PY{n}{sort}\PY{p}{(}\PY{n}{key}\PY{o}{=}\PY{n}{pairs}\PY{p}{[}\PY{l+m+mi}{1}\PY{p}{]}\PY{p}{)}    \PY{c+c1}{\PYZsh{}按里层list第二个元素排序 也就是 按照one two three four的首字母排序 下边list.sort()有进一步解释}
        \PY{n}{pairs}
\end{Verbatim}


    \begin{Verbatim}[commandchars=\\\{\}]

        ---------------------------------------------------------------------------

        TypeError                                 Traceback (most recent call last)

        <ipython-input-9-8f846dbd2cf1> in <module>()
          1 pairs = [(1, 'one'), (2, 'two'), (3, 'three'), (4, 'four')]
    ----> 2 pairs.sort(key=pairs[1])    \#按里层list第二个元素排序 也就是 按照one two three four的首字母排序 下边list.sort()有进一步解释
          3 pairs
    

        TypeError: 'tuple' object is not callable

    \end{Verbatim}

    \begin{enumerate}
\def\labelenumi{\arabic{enumi}.}
\setcounter{enumi}{9}
\tightlist
\item
  list.reverse()
\end{enumerate}

Reverse the elements of the list in place.

    \begin{enumerate}
\def\labelenumi{\arabic{enumi}.}
\setcounter{enumi}{10}
\tightlist
\item
  list.copy()
\end{enumerate}

Return a shallow copy of the list. Equivalent to \textbf{a{[}:{]}}.

    Examples that uses most of the list methods:

    \begin{Verbatim}[commandchars=\\\{\}]
{\color{incolor}In [{\color{incolor}10}]:} \PY{n}{fruits} \PY{o}{=} \PY{p}{[}\PY{l+s+s1}{\PYZsq{}}\PY{l+s+s1}{orrange}\PY{l+s+s1}{\PYZsq{}}\PY{p}{,} \PY{l+s+s1}{\PYZsq{}}\PY{l+s+s1}{apple}\PY{l+s+s1}{\PYZsq{}}\PY{p}{,} \PY{l+s+s1}{\PYZsq{}}\PY{l+s+s1}{pear}\PY{l+s+s1}{\PYZsq{}}\PY{p}{,} \PY{l+s+s1}{\PYZsq{}}\PY{l+s+s1}{banana}\PY{l+s+s1}{\PYZsq{}}\PY{p}{,} \PY{l+s+s1}{\PYZsq{}}\PY{l+s+s1}{kiwi}\PY{l+s+s1}{\PYZsq{}}\PY{p}{,} \PY{l+s+s1}{\PYZsq{}}\PY{l+s+s1}{apple}\PY{l+s+s1}{\PYZsq{}}\PY{p}{,} \PY{l+s+s1}{\PYZsq{}}\PY{l+s+s1}{banana}\PY{l+s+s1}{\PYZsq{}}\PY{p}{]}
         \PY{n}{fruits}\PY{o}{.}\PY{n}{count}\PY{p}{(}\PY{l+s+s1}{\PYZsq{}}\PY{l+s+s1}{apple}\PY{l+s+s1}{\PYZsq{}}\PY{p}{)}
\end{Verbatim}


\begin{Verbatim}[commandchars=\\\{\}]
{\color{outcolor}Out[{\color{outcolor}10}]:} 2
\end{Verbatim}
            
    \begin{Verbatim}[commandchars=\\\{\}]
{\color{incolor}In [{\color{incolor}11}]:} \PY{n}{fruits}\PY{o}{.}\PY{n}{count}\PY{p}{(}\PY{l+s+s1}{\PYZsq{}}\PY{l+s+s1}{tangerine}\PY{l+s+s1}{\PYZsq{}}\PY{p}{)}
\end{Verbatim}


\begin{Verbatim}[commandchars=\\\{\}]
{\color{outcolor}Out[{\color{outcolor}11}]:} 0
\end{Verbatim}
            
    \begin{Verbatim}[commandchars=\\\{\}]
{\color{incolor}In [{\color{incolor}12}]:} \PY{n}{fruits}\PY{o}{.}\PY{n}{index}\PY{p}{(}\PY{l+s+s1}{\PYZsq{}}\PY{l+s+s1}{banana}\PY{l+s+s1}{\PYZsq{}}\PY{p}{)}
\end{Verbatim}


\begin{Verbatim}[commandchars=\\\{\}]
{\color{outcolor}Out[{\color{outcolor}12}]:} 3
\end{Verbatim}
            
    \begin{Verbatim}[commandchars=\\\{\}]
{\color{incolor}In [{\color{incolor}13}]:} \PY{n}{fruits}\PY{o}{.}\PY{n}{index}\PY{p}{(}\PY{l+s+s1}{\PYZsq{}}\PY{l+s+s1}{banana}\PY{l+s+s1}{\PYZsq{}}\PY{p}{,} \PY{l+m+mi}{4}\PY{p}{)}  \PY{c+c1}{\PYZsh{} Find next banana starting at position 4}
\end{Verbatim}


\begin{Verbatim}[commandchars=\\\{\}]
{\color{outcolor}Out[{\color{outcolor}13}]:} 6
\end{Verbatim}
            
    \begin{Verbatim}[commandchars=\\\{\}]
{\color{incolor}In [{\color{incolor}14}]:} \PY{n}{fruits}\PY{o}{.}\PY{n}{reverse}\PY{p}{(}\PY{p}{)}
\end{Verbatim}


    \begin{Verbatim}[commandchars=\\\{\}]
{\color{incolor}In [{\color{incolor}15}]:} \PY{n}{fruits}
\end{Verbatim}


\begin{Verbatim}[commandchars=\\\{\}]
{\color{outcolor}Out[{\color{outcolor}15}]:} ['banana', 'apple', 'kiwi', 'banana', 'pear', 'apple', 'orrange']
\end{Verbatim}
            
    \begin{Verbatim}[commandchars=\\\{\}]
{\color{incolor}In [{\color{incolor}16}]:} \PY{n}{fruits}\PY{o}{.}\PY{n}{append}\PY{p}{(}\PY{l+s+s1}{\PYZsq{}}\PY{l+s+s1}{grape}\PY{l+s+s1}{\PYZsq{}}\PY{p}{)}
\end{Verbatim}


    \begin{Verbatim}[commandchars=\\\{\}]
{\color{incolor}In [{\color{incolor}17}]:} \PY{n}{fruits}
\end{Verbatim}


\begin{Verbatim}[commandchars=\\\{\}]
{\color{outcolor}Out[{\color{outcolor}17}]:} ['banana', 'apple', 'kiwi', 'banana', 'pear', 'apple', 'orrange', 'grape']
\end{Verbatim}
            
    \begin{Verbatim}[commandchars=\\\{\}]
{\color{incolor}In [{\color{incolor}18}]:} \PY{n}{fruits}\PY{o}{.}\PY{n}{sort}\PY{p}{(}\PY{p}{)}
\end{Verbatim}


    \begin{Verbatim}[commandchars=\\\{\}]
{\color{incolor}In [{\color{incolor}19}]:} \PY{n}{fruits}
\end{Verbatim}


\begin{Verbatim}[commandchars=\\\{\}]
{\color{outcolor}Out[{\color{outcolor}19}]:} ['apple', 'apple', 'banana', 'banana', 'grape', 'kiwi', 'orrange', 'pear']
\end{Verbatim}
            
    \begin{Verbatim}[commandchars=\\\{\}]
{\color{incolor}In [{\color{incolor}20}]:} \PY{n}{fruits}\PY{o}{.}\PY{n}{pop}\PY{p}{(}\PY{p}{)}
\end{Verbatim}


\begin{Verbatim}[commandchars=\\\{\}]
{\color{outcolor}Out[{\color{outcolor}20}]:} 'pear'
\end{Verbatim}
            
    \paragraph{Using Lists as Stacks}\label{using-lists-as-stacks}

    \begin{itemize}
\tightlist
\item
  The list methods make it very easy to use a list as a stack, where the
  last element added is the first element retrieved (last-in,
  first-out).
\item
  To add an item to the top of the stack, use \textbf{append()}. To
  retrieve an item from the top of the stack, use \textbf{pop()} without
  an explicit index. For example:
\end{itemize}

    \begin{Verbatim}[commandchars=\\\{\}]
{\color{incolor}In [{\color{incolor}21}]:} \PY{n}{stack} \PY{o}{=} \PY{p}{[}\PY{l+m+mi}{3}\PY{p}{,} \PY{l+m+mi}{4}\PY{p}{,} \PY{l+m+mi}{5}\PY{p}{]}
         \PY{n}{stack}\PY{o}{.}\PY{n}{append}\PY{p}{(}\PY{l+m+mi}{6}\PY{p}{)}
         \PY{n}{stack}\PY{o}{.}\PY{n}{append}\PY{p}{(}\PY{l+m+mi}{7}\PY{p}{)}
         \PY{n}{stack}
\end{Verbatim}


\begin{Verbatim}[commandchars=\\\{\}]
{\color{outcolor}Out[{\color{outcolor}21}]:} [3, 4, 5, 6, 7]
\end{Verbatim}
            
    \begin{Verbatim}[commandchars=\\\{\}]
{\color{incolor}In [{\color{incolor}22}]:} \PY{n}{stack}\PY{o}{.}\PY{n}{pop}\PY{p}{(}\PY{p}{)}
\end{Verbatim}


\begin{Verbatim}[commandchars=\\\{\}]
{\color{outcolor}Out[{\color{outcolor}22}]:} 7
\end{Verbatim}
            
    \begin{Verbatim}[commandchars=\\\{\}]
{\color{incolor}In [{\color{incolor}23}]:} \PY{n}{stack}
\end{Verbatim}


\begin{Verbatim}[commandchars=\\\{\}]
{\color{outcolor}Out[{\color{outcolor}23}]:} [3, 4, 5, 6]
\end{Verbatim}
            
    \begin{Verbatim}[commandchars=\\\{\}]
{\color{incolor}In [{\color{incolor}24}]:} \PY{n}{stack}\PY{o}{.}\PY{n}{pop}\PY{p}{(}\PY{p}{)}
\end{Verbatim}


\begin{Verbatim}[commandchars=\\\{\}]
{\color{outcolor}Out[{\color{outcolor}24}]:} 6
\end{Verbatim}
            
    \begin{Verbatim}[commandchars=\\\{\}]
{\color{incolor}In [{\color{incolor}25}]:} \PY{n}{stack}
\end{Verbatim}


\begin{Verbatim}[commandchars=\\\{\}]
{\color{outcolor}Out[{\color{outcolor}25}]:} [3, 4, 5]
\end{Verbatim}
            
    \begin{Verbatim}[commandchars=\\\{\}]
{\color{incolor}In [{\color{incolor}26}]:} \PY{n}{stack}\PY{o}{.}\PY{n}{pop}\PY{p}{(}\PY{p}{)}
\end{Verbatim}


\begin{Verbatim}[commandchars=\\\{\}]
{\color{outcolor}Out[{\color{outcolor}26}]:} 5
\end{Verbatim}
            
    \begin{Verbatim}[commandchars=\\\{\}]
{\color{incolor}In [{\color{incolor}27}]:} \PY{n}{stack}
\end{Verbatim}


\begin{Verbatim}[commandchars=\\\{\}]
{\color{outcolor}Out[{\color{outcolor}27}]:} [3, 4]
\end{Verbatim}
            
    \paragraph{Using Lists as Queues}\label{using-lists-as-queues}

    As queues, the first element added is the first element retrieved
(first-in, first-out); however, \textbf{lists are not efficient for this
purpose}.Because all of the other elements have to be shifted by one.To
implement a queue, use a \textbf{collections.deque} which was designed
to have fast appends and pops from both ends. For example:

    \begin{Verbatim}[commandchars=\\\{\}]
{\color{incolor}In [{\color{incolor}30}]:} \PY{k+kn}{from} \PY{n+nn}{collections} \PY{k}{import} \PY{n}{deque}
         \PY{n}{queue} \PY{o}{=} \PY{n}{deque}\PY{p}{(}\PY{p}{[}\PY{l+s+s2}{\PYZdq{}}\PY{l+s+s2}{Eric}\PY{l+s+s2}{\PYZdq{}}\PY{p}{,} \PY{l+s+s2}{\PYZdq{}}\PY{l+s+s2}{John}\PY{l+s+s2}{\PYZdq{}}\PY{p}{,} \PY{l+s+s2}{\PYZdq{}}\PY{l+s+s2}{Michael}\PY{l+s+s2}{\PYZdq{}}\PY{p}{]}\PY{p}{)}
         \PY{n}{queue}\PY{o}{.}\PY{n}{append}\PY{p}{(}\PY{l+s+s2}{\PYZdq{}}\PY{l+s+s2}{Terry}\PY{l+s+s2}{\PYZdq{}}\PY{p}{)}
         \PY{n}{queue}\PY{o}{.}\PY{n}{append}\PY{p}{(}\PY{l+s+s2}{\PYZdq{}}\PY{l+s+s2}{Graham}\PY{l+s+s2}{\PYZdq{}}\PY{p}{)}
         \PY{n}{queue}
\end{Verbatim}


\begin{Verbatim}[commandchars=\\\{\}]
{\color{outcolor}Out[{\color{outcolor}30}]:} deque(['Eric', 'John', 'Michael', 'Terry', 'Graham'])
\end{Verbatim}
            
    \begin{Verbatim}[commandchars=\\\{\}]
{\color{incolor}In [{\color{incolor}31}]:} \PY{n}{queue}\PY{o}{.}\PY{n}{popleft}\PY{p}{(}\PY{p}{)}    \PY{c+c1}{\PYZsh{} no popright() function}
\end{Verbatim}


\begin{Verbatim}[commandchars=\\\{\}]
{\color{outcolor}Out[{\color{outcolor}31}]:} 'Eric'
\end{Verbatim}
            
    \begin{Verbatim}[commandchars=\\\{\}]
{\color{incolor}In [{\color{incolor}29}]:} \PY{n}{queue}\PY{o}{.}\PY{n}{popleft}\PY{p}{(}\PY{p}{)}
\end{Verbatim}


\begin{Verbatim}[commandchars=\\\{\}]
{\color{outcolor}Out[{\color{outcolor}29}]:} 'John'
\end{Verbatim}
            
    \begin{Verbatim}[commandchars=\\\{\}]
{\color{incolor}In [{\color{incolor}32}]:} \PY{n}{queue}
\end{Verbatim}


\begin{Verbatim}[commandchars=\\\{\}]
{\color{outcolor}Out[{\color{outcolor}32}]:} deque(['John', 'Michael', 'Terry', 'Graham'])
\end{Verbatim}
            
    \paragraph{List Comprehensions}\label{list-comprehensions}


    % Add a bibliography block to the postdoc
    
    
    
    \end{document}
